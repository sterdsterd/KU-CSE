% !TeX program = xelatex
\documentclass[runningheads]{llncs}
\usepackage[paperheight=295mm,paperwidth=210mm]{geometry}
\usepackage{graphicx}
\usepackage{wrapfig}
\usepackage{import}
\usepackage{kotex}
\usepackage[dvipsnames]{xcolor}
\usepackage{fancyvrb} %
\usepackage{listings}
\usepackage{tabularx}
\usepackage{underscore}
\usepackage{multicol}
\usepackage{enumitem}
\usepackage{subcaption}
\usepackage[numbers,square,super]{natbib}
\usepackage{mathptmx} % Times New Roman
\usepackage{amsmath}
\usepackage{amssymb}
\usepackage{framed}
\usepackage{etoolbox}
\usepackage{cancel}
\usepackage{physics}
\usepackage{tikz}
\usepackage{parskip}
\usepackage{enumerate}
\usepackage{minted}
\usepackage{inconsolata}
\usepackage{makecell}
\usepackage{slashed}
\usepackage{nicematrix}
\usetikzlibrary{calc, angles, quotes, graphs, positioning, arrows}

\setcounter{tocdepth}{2}

\colorlet{shadecolor}{gray!30}

\newcommand\enclosebox[2]{%
  \BeforeBeginEnvironment{#1}{\begin{#2}}%
  \AfterEndEnvironment{#1}{\end{#2}}%
}

\enclosebox{theorem}{oframed}
\enclosebox{definition}{leftbar}

\newcommand{\divides}{\bigm|}
\newcommand{\ndivides}{%
  \mathrel{\mkern.5mu % small adjustment
    % superimpose \nmid to \big|
    \ooalign{\hidewidth$\big|$\hidewidth\cr$\nmid$\cr}%
  }%
}
\newcommand{\ord}{\operatorname{\mathrm{ord}}}
\newcommand{\ind}{\operatorname{\mathrm{ind}}}
\newcommand{\legendre}[2]{\left(\frac{#1}{#2}\right)}
\setmainfont{Times New Roman}
\setmainhangulfont{Nanum Myeongjo}
\setmonofont{SF Mono}
\setlength{\parindent}{1em}
\setlength{\parskip}{0pt}
\linespread{1.2}
%\renewcommand{\arraystretch}{1.5}
\setlength{\tabcolsep}{0.5em}%
\newenvironment{Figure}
  {\par\medskip\noindent\minipage{\linewidth}}
  {\endminipage\par\medskip}
\newcommand{\translation}[1]{\textsuperscript{#1}}

\makeatletter
\renewcommand\NAT@citesuper[3]{\ifNAT@swa
\if*#2*\else#2\NAT@spacechar\fi
\unskip\kern\p@\textsuperscript{\NAT@@open#1\if*#3*\else,\NAT@spacechar#3\fi\NAT@@close}%
   \else #1\fi\endgroup}
\makeatother

\let\oldtabular\tabular% Store a copy of \tabular
\let\endoldtabular\endtabular% Store a copy of \endtabular
\renewenvironment{tabular}[2][\arraystretch]
  {\edef\arraystretch{#1}% Update \arraystretch
   \oldtabular{#2}}% \begin{tabular}[<stretch>]{<col spec>}
  {\endoldtabular}% \end{tabular}

\begin{document}

\title{Discrete Mathematics (0034)\newline\space Lecture Notes}
\author{Yulwon Rhee (202211342)}
\institute{Department of Computer Science and Engineering, Konkuk University}

\maketitle
\section{Week 1}
\subsection{논리와 명제}
논리 : 사고의 규칙\\
명제 논리(Propositional Logic): T/F 판별 가능한 문장 or 수식\\
술어 논리(Predicate Logic): 변수 포함 명제\\
단순 명제(simple Proposition): 하나의 문장 or 수식으로 구성된 명제\\
합성 명제(Composition Proposition): 단순 명제들이 논리 연산자로 연결\\\\
항진 명제(Tautology): 항상 T인 합성 명제\\
모순 명제(Contradiction): 항상 F인 합성 명제\\
\subsection{논리 연산자}
\begin{table}[]
    \begin{minipage}{\linewidth}
        \caption {Logical Operators}
        \centering
        \begin{tabular}{c|c}
        연산자      & 기호\\
        \Xhline{3\arrayrulewidth}
        부정(NOT)  & $\sim$\\
        \hline
        논리곱(AND) & $\land$\\
        \hline
        논리합(OR)  & $\lor$\\
        \hline
        배타적 논리합(XOR)&$\oplus$\\
        \hline
        조건(if then)&$\to$\\
        \hline
        쌍방 조건(iff)&$\leftrightarrow$
        \end{tabular}
    \end{minipage}
\end{table}
\begin{table}[]
    \caption {Truth Table for the XOR, Implication and Biconditional Proposition}
    \begin{minipage}{.33\linewidth}
        \centering
        \begin{tabular}{cc|c}
        $p$&$q$&$p\oplus q$\\
        \Xhline{3\arrayrulewidth}
        T&T&F\\
        T&F&T\\
        F&T&T\\
        F&F&F
        \end{tabular}
    \end{minipage}
    \begin{minipage}{.33\linewidth}
        \centering
        \begin{tabular}{cc|c}
        $p$&$q$&$p\to q$\\
        \Xhline{3\arrayrulewidth}
        T&T&T\\
        T&F&F\\
        F&T&T\\
        F&F&T
        \end{tabular}
    \end{minipage}
    \begin{minipage}{.33\linewidth}
        \centering
        \begin{tabular}{cc|c}
        $p$&$q$&$p\leftrightarrow q$\\
        \Xhline{3\arrayrulewidth}
        T&T&T\\
        T&F&F\\
        F&T&F\\
        F&F&T
        \end{tabular}
    \end{minipage}
\end{table}
\newpage
\begin{itemize}
    \item $p$이면 $q$이다. (if $p$ then $q$, $q$ when $p$, $p$ only if $q$)
    \item $p$는 $q$의 충분조건이다. ($p$ is sufficient for $q$)
    \item $q$는 $p$의 필요조건이다. ($q$ is necessary for $p$)
    \item $p$는 $q$를 함축한다. ($p$ implies $q$)
\end{itemize}

\subsection{논리 연산자 우선순위}
$$\sim\quad>\quad\land\quad>\quad\lor\quad>\quad\to\quad>\quad\leftrightarrow$$
\subsection{논리 연산: 상호 관계}
\begin{table}[]
    \caption {$p\to q$에 대하여}
    \centering
    \begin{tabular}{c|c}
        역(Converse)&$q\to p$\\
        \hline
        이(Inverse)&$\sim p \to \sim q$\\
        \hline
        대우(Contrapositive)&$\sim q \to \sim p$
    \end{tabular}
\end{table}
\subsection{예제 풀이}
e.g.) You cannot ride the rollercoaster if you are under 4 ft. tall unless you are older than 16 years old.\\
$p$ : You can ride the rollercoaster\\
$q$ : You are under 4 ft. tall\\
$r$ : You are older than 16 years old\\
$\sim p$ if $q$ unless $r$\\$\Rightarrow$ $\sim p$ if $q$ if $\sim r$\\$\Rightarrow$ $\sim r \to (q \to \sim p)\\\equiv (\sim r \land q) \to \sim p$\\
\newpage
\section{Week 2}
\subsection{비트 연산(Bit Operations)}
\begin{table}[]
    \caption {Logical Operators}
    \centering
    \begin{tabular}{c|c|c|c|c|c|c}
    Name      & NOT & AND & OR & XOR & Implies & Iff\\
    \Xhline{3\arrayrulewidth}
    Propositional Logic & $\sim$ & $\land$ & $\lor$ & $\oplus$ & $\to$ & $\leftrightarrow$\\
    \hline
    Boolean Algebra & $\overline{p}$ & $p\cdot q$ & $+$ & $\oplus$ & &\\
    \hline
    C/C++/Java(Wordwise) & \texttt{!} & \texttt{\&\&} & \texttt{||} & \texttt{!=} & & \texttt{==}\\
    \hline
    C/C++/Java(Bitwise) & \texttt{ \~} & \texttt{\&} & \texttt{|} & \texttt{ \^} & &\\
    \end{tabular}
\end{table}
\subsection{논리적 동치 관계}
$p\leftrightarrow q$가 항진 명제 $\to$ $p, q$는 논리적 동치, $p \equiv q$ 또는 $p \Leftrightarrow q$\\
\begin{table}[]
    \caption {논리적 동치 관계}
    \centering
    \begin{tabular}{c|c}
        법칙 이름&동치 관계\\
        \Xhline{3\arrayrulewidth}
        결합 법칙&\begin{tabular}{@{}c@{}}$(p\lor q)\lor r \Leftrightarrow p \lor (q \lor r)$\\$(p\land q)\land r \Leftrightarrow p \land (q \land r)$\end{tabular}\\
        \hline
        홀수 법칙&\begin{tabular}{@{}c@{}}$p\lor(p\land q) \Leftrightarrow p$\\$p\land(p\lor q) \Leftrightarrow p$\end{tabular}\\
        \hline
        드 모르간 법칙&\begin{tabular}{@{}c@{}}$\sim(p\lor q)\equiv(\sim p)\land(\sim q)$\\$\sim(p\land q)\equiv(\sim p)\lor(\sim q)$\end{tabular}\\
        \hline
        조건 법칙& $p\to q\Leftrightarrow\sim p\lor q$\\
        \hline
        대우 법칙& $p\to q\Leftrightarrow\sim q\to\sim p$
    \end{tabular}
\end{table}
\newpage
\subsection{술어 논리(Predicate Logic)}
$p(x)=x$에 대한 명제술어
\subsection{술어 한정자(Predicate Quantifier)}
$\forall$: 모든\\
$\exists$: 어떤\\
괄호를 이용하여 모순이 없도록 범위 지정 필요\\
제한된 정의역 표현: 변수가 만족해야 하는 조건이 한정기호 다음에 표기\\
e.g.) $$\forall x<0(x^2>0)\text{, 정의역은 실수 }\Longrightarrow \forall x (x < 0 \rightarrow x^2>0)$$
$$\exists x>0(x^2=2)\text{, 정의역은 실수 }\Longrightarrow \exists x(x>0 \land x^2=2)$$
\\
부정:
$$\sim \forall(p(x)) \equiv \exists x(\sim p(x))$$
$$\sim (\exists x p(x)) \equiv \forall x(\sim p(x))$$

\subsection{중첩 한정자}
$\forall x \exists y\ P(x, y)$: For every $x$, there is a $y$ for which $P(x, y)$ is true.\\
$\exists x \forall y\ P(x, y)$: There is an $x$ for which $P(x, y)$ is true for every $y$.

\subsection{예제 풀이}
1. If a user is active at least one network link will be available.

$A(u)$ : User $u$ is active.

$S(n, x)$ : Network link $n$ is in status $x$.

$\exists u\ A(u) \to \exists n\ S(n, available)$\\\\
2. Everyone has exactly one best friend.
\begin{itemize}[label=$\to$]
    \item For every person $x$, person $x$ has exactly one best friend.
    \item 'Exactly one' means that
    \begin{itemize}
        \item[1.] There is a person $y$ who is the best friend of $x$.
        \item[2.] For every person $z$, if person $z$ is not $y$, then $z$ is not the best friend of $x$.
    \end{itemize}
\end{itemize}

$B(x, y)$: $y$ is the best friend of $x$.

$\forall x \exists y (B(x, y))$, $y$의 조건: $\forall z((z\neq y) \to \sim B(x, z))$\\\\
3. There is somebody whom everybody loves.

$\exists y\forall x L(x, y)$ ($\exists y$와 $\forall x$ 순서 유의!)\\\\
4. Nobody loves everybody

$\sim \exists x \forall y\ L(x, y) \equiv \forall x \exists y\ \sim L(x, y)$
\section{Week 3}
\subsection{추론}
(연역적) 추론(Argument): 주어진 명제 $p_n$을 바탕으로 새로운 명제 $q$를 유도

$p_n$: 전제(Premise), 가정(Hypothesis)

$q$: 결론(Conclusion)\\
유효 추론(Valid Argument): 전제 T, 결론 T\\
허위 추론(Fallacious Argument): 결론 F
\begin{table}[H]
    \caption {논리적 추론 법칙}
    \centering
    \begin{tabular}{c|c}
        법칙 이름&추론 법칙\\
        \Xhline{3\arrayrulewidth}
        긍정 법칙*&$p,\ p\to q\vdash q$\\
        \hline
        부정 법칙*&$\sim q,\ p \to q\vdash \sim p$\\
        \hline
        조건 삼단 법칙*&$p \to q,\ q \to r\vdash p \to r$\\
        \hline
        선언 삼단 법칙&$p \lor q,\ \sim p\vdash q$\\
        \hline
        양도 법칙&$(p\to q)\land(r\to s),\ p\lor r\vdash (q\lor s)$\\
        \hline
        파괴적 법칙&$(p\to q)\land(r\to s),\ \sim q \lor \sim s\vdash \sim p \lor \sim r$\\
        \hline
        선접 법칙&$p \vdash p \lor q$\\
        \hline
        분리 법칙&$p \land q\vdash p$\\
        \hline
        연접 법칙&$p,\ q\vdash p \land q$
    \end{tabular}
    \\
    \bigskip
    * 가장 많이 사용되고 잘 알려진 3가지 법칙
\end{table}
\subsection{대치 vs 추론}
대치의 공식은 합성 명제 전체 또는 한 부분에 적용 가능\\
추론의 법칙은 합성 명제의 주 연산자에 사용

\subsection{증명법: 한정기호를 사용한 명제의 추론규칙}
전칭 예시화(Universal Instantiation): $\forall x P(x) \to \exists c P(c)$\\
전칭 일반화(Universal Generalisation): $P(c)\text{ for an arbitrary }c \to \forall x P(x)$\\
존재 예시화(Existential Instantiation): $\exists x P(x) \to P(c)$ for some $c$ ($c$가 적어도 하나 존재)\\
존재 일반화(Existential Generalisation): $P(c)$ for some $c \to \exists x P(x)$

\subsection{예제 풀이}
1. There is someone in this class who has been to France

Everyine who goes to France visits the Louvre.

Therefore, someone in this class has visited the Louvre.
\subsubsection{Solution.}$$$$
$x$: 사람\\
$C(x): x$ is in this class.\\
$F(x): x$ has been to France.\\
$L(x): x$ visits to Louvre.\\\\
$\exists x(C(x) \land F(x)),\ \forall x(F(x) \to L(x))\vdash\exists x(C(x)\land L(x))$\\\\
Some $c$, $C(c)\land F(c)$: T (존재 예시화)\\
$\forall x \to c \in x$, $F(c) \to L(c)$: T (전칭 예시화)\\
$C(c) \land F(c) \vdash C(c), F(c)$\\
$F(c),\ F(c)\to L(c) \vdash L(c)$\\
$C(c),\ L(c)\to C(c) \land L(c)$\\
$C(c)\land L(c) \to \exists x (C(x) \land L(x))$ (존재 일반화)

\subsection{정리의 증명}
정의: 논의 대상 보편화 위해 사용 용어 or 기호 의미를 확실히 규명한 문장 or 식

e.g.) 한 내각의 크기가 직각인 삼각형은 직각삼각형, 명제는 T/F 판별 가능한 문장 or 수식\\
공리: 별도 증명 없이 T로 이용되는 명제

e.g) $p$가 참이면 $p\lor q$도 참, $a=b$면, $a+c=b+c$\\
정리: 공리, 정의로 T가 확인된 명제\\
증명: 공리, 정의, 정리로 명제가 T임을 확인하는 과정

\subsection{증명 방법}
$p\to q$ 증명: $p, q$ 모두 T or $p$ 무조건 거짓\\
직접 증명법: $p\to q$ 직접 증명\\
간접 증명법: 동치로 $p \to q$ 변환하여 증명. 대우 증명법, 모순 증명법, 반례 증명법, 존재 증명법\\
기타 증명법: 수학적 귀납법

\subsection{수학적 귀납법}
연역법(Deduction): 사실(Fact), 공리(Axiom)에 입각해 추론(Inference)을 통해 새로운 사실 도출\\
귀납법(Induction): 관찰, 실험에 기반한 가설을 귀납 추론을 통해 일반적인 규칙으로 입증

\section{Week 4}
\subsection{직접 증명법(Direct Proof)}
$p \to q$가 T 증명

\subsection{모순 증명법(귀류법, Contradiction Proof)}
주어진 문제의 명제 부정 후 논리 전개\\
\begin{flushleft}
    $\begin{aligned}
        \sim (p \land (\sim q)) &\equiv \sim p \lor \sim(\sim q)\\
        &\equiv \sim p \lor \sim q\\
        &\equiv \sim p \lor q\\
        &\equiv p \to q
    \end{aligned}$
\end{flushleft}$$$$
$p \land (\sim q)$가 T라고 하고, 모순 유도 시 원래 명제 T

\subsection{대우 증명법(Contrapositive Proof)}
$p \to q \equiv \sim q \to \sim p$에서, $\sim q \to \sim p$가 T 증명

\subsection{존재 증명법(Existence Proof)}
$\exists x$ such that $p(x)$ 증명

\subsection{반례 증명법(Counter-Example Proof)}
반례를 통해 증명\\
$\forall x p(x)$가 F임을 보이기 위해 $\sim \forall x p(x) \equiv \exists x \sim p(x)$에서  $p(x)$가 F인 $x$ 적어도 하나 존재

\subsection{필요충분조건 증명법(Iff Proof)}
$p \to q,\ q\to p$ 증명 $\Rightarrow p \leftrightarrow q$ 증명

\newpage
\section{Week 5}
\subsection{집합}
Cardinality: 원소 개수\\
부분 집합(Subset): $A$의 모든 원소가 $B$의 원소에 속할 때, $A\subseteq B$. 부분 집합이 아닐 때, $A \nsubseteq B$\\
진부분 집합(Proper Subset): $A \subseteq B$, $A \neq B \Longrightarrow A\subset B$. 진부분 집합이 아닐 때, $A \not\subset B$\\
멱집합(Power Set): 모든 부분 집합을 원소로 가지는 집합 $= P(S) = 2^S$. $|P(S)|=2^{|S|}$
\subsection{부분 집합의 성질}
\begin{itemize}
    \item $\forall P,\ P \subseteq P$
    \item $\forall P,\ \varnothing \subseteq P$
\end{itemize}
\subsection{집합의 연산}
\begin{table}[]
    \caption {Set Operators}
    \centering
    \begin{tabular}{c|c}
    연산      & 기호\\
    \Xhline{3\arrayrulewidth}
    합집합  & $A \cup B$\\
    \hline
    교집합 & $A \cap B$\\
    \hline
    차집합 &$A-B$\\
    \hline
    대칭 차집합&$A\oplus B$\\
    \hline
    곱집합&$A\times B$
    \end{tabular}
\end{table}$$$$
서로소: $A \cap B = \varnothing$\\
곱집합(Cartesian Product): $a\in A,\ b \in B,\ (a, b)$인 모든 순서쌍의 집합

e.g) $A={1, 2, 3},\ B={a, b, c}$라 할 때, $A\times B = {(1, a), (1, b), (1, c), (2, a), (2, b), (2, c), (3, a), (3, b), (3, c)}$\\
드 모르간 법칙: $\overline{(A\cup B)} = \overline{A} \cap \overline{B}$, $\overline{(A \cap B)} = \overline{A} \cap \overline{B}$

\subsection{집합의 분할}
분할(Partition): $\exists S \neq \varnothing (\pi = \left\{A_1, A_2, \cdots, A_i, \cdots, A_k\right\})$
\begin{itemize}
    \item[1.] $i=1, \cdots, k$에 대하여, $A_i \subseteq S\ (S \neq \varnothing)$
    \item[2.] $S=A_1\cup A_2\cup \cdots \cup A_k$
    \item[3.] $i\neq j \to A_i \cap A_j = \varnothing$
\end{itemize}

c.f.) $A_i = $ 분할의 블록

\section{Week 6}
\subsection{보수}
$r$진수 정수 $N$에서 $r-1$의 보수: $(r^n-1) - N$\\
$r$진수 정수 $N$에서 $r$의 보수: $(r^n) - N = (r-1\text{의 보수}) + 1$

\subsection{부호화 절대치 표현}
\begin{itemize}
    \item 연산 결과가 정확하지 않음
    \item 0의 표현이 2가지
\end{itemize}
\subsection{1의 보수 표현}
\begin{itemize}
    \item 연산 결과는 정확하지만 (초과 비트를 더해줄 때)
    \item 0의 표현이 2가지
\end{itemize}
\subsection{2의 보수 표현}
\begin{itemize}
    \item 연산 결과가 정확함
    \item 0의 표현이 1가지
    \item 음수 값 하나 더 표현 가능 (0의 표현이 하나 줄어들어서)
\end{itemize}
\subsection{초과 비트 발생 시}
\begin{itemize}
    \item 1의 보수: 초과 비트를 덧셈
    \item 2의 보수: 무시
\end{itemize}

\newpage
\section{Week 7}
\subsection{행렬}
대각합(Trace): 대각성분의 합. $\mathrm{tr}(A) = \mathrm{trace}(A)$\\
교대 행렬(Skewed-Symmetric Matrix): $A=-A^T$

\subsection{행렬식}
$\mathrm{Det}(A) = |A|$\\\\
let $A = \begin{bmatrix}
    a_{11} & a_{12}\\
    a_{21} & a_{22}
\end{bmatrix}$\\
$\mathrm{Det}(A) = \begin{vmatrix}
    a_{11} & a_{12}\\
    a_{21} & a_{22}
\end{vmatrix} = a_{11}a_{22}-a_{12}a_{21}$\\\\\\
let $B = \begin{bmatrix}
    b_{11}&b_{12}&b_{13}\\
    b_{21}&b_{22}&b_{23}\\
    b_{31}&b_{32}&b_{33}
\end{bmatrix}$\\
$\mathrm{Det}(B) = \begin{vmatrix}
    b_{11}&b_{12}&b_{13}\\
    b_{21}&b_{22}&b_{23}\\
    b_{31}&b_{32}&b_{33}
\end{vmatrix} = b_{11}b_{22}b_{33}+b_{12}b_{23}b_{31}+b_{13}b_{21}b_{31}-b_{11}b_{23}b_{32}-b_{12}-b_{21}-b_{33}-b_{13}b_{22}b_{31}$\\\\\\
정칙 행렬(Non-Singular Matrix): $\mathrm{Det}(A) \neq 0$\\
특이 행렬(Singular Matrix): $\mathrm{Det}(A) = 0$\\
\subsection{행렬식의 성질}
\begin{itemize}
    \item $n \times n$ 행렬 $A$에서 임의의 두 행 또는 열이 같으면 $\mathrm{Det}(A) = 0$
    \item $n \times n$ 행렬 $A$에서 임의의 두 행 또는 열을 바꾸어서 만든 행렬 $B$에서 $\mathrm{Det}(B) = -\mathrm{Det}(A)$
    \item $n \times n$ 행렬 $A$에서 임의의 행 또는 열의 모든 원소가 $0$이면 $\mathrm{Det}(A) = 0$
    \item $\mathrm{Det}(A) = \mathrm{Det}(A^T)$
    \item $\mathrm{Det}(AB) = \mathrm{Det}(A) \cdot \mathrm{Det}(B)$
    \item $\mathrm{Det}(kA) = k\mathrm{Det}(A)$
\end{itemize}$$$$
가역적(Nonsingular, Invertible): $A, B$가 정칙 행렬. $AB=BA=I$인 경우
\end{document}