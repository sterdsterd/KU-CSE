% !TeX program = xelatex
\documentclass[runningheads]{llncs}
\usepackage[paperheight=295mm,paperwidth=210mm]{geometry}
\usepackage{graphicx}
\usepackage{wrapfig}
\usepackage{import}
\usepackage{kotex}
\usepackage[dvipsnames]{xcolor}
\usepackage{fancyvrb} %
\usepackage{listings}
\usepackage{tabularx}
\usepackage{underscore}
\usepackage{multicol}
\usepackage{enumitem}
\usepackage{subcaption}
\usepackage[numbers,square,super]{natbib}
\usepackage{mathptmx} % Times New Roman
\usepackage{amsmath}
\usepackage{amssymb}
\usepackage{framed}
\usepackage{etoolbox}
\usepackage{cancel}
\usepackage{physics}
\usepackage{tikz}
\usepackage{parskip}
\usepackage{enumerate}
\usepackage{minted}
\usepackage{inconsolata}
\usepackage{makecell}
\usepackage{slashed}
\usepackage{nicematrix}
\usetikzlibrary{calc, angles, quotes, graphs, positioning, arrows}

\setcounter{tocdepth}{2}

\colorlet{shadecolor}{gray!30}

\newcommand\enclosebox[2]{%
  \BeforeBeginEnvironment{#1}{\begin{#2}}%
  \AfterEndEnvironment{#1}{\end{#2}}%
}

\enclosebox{theorem}{oframed}
\enclosebox{definition}{leftbar}

\newcommand{\divides}{\bigm|}
\newcommand{\ndivides}{%
  \mathrel{\mkern.5mu % small adjustment
    % superimpose \nmid to \big|
    \ooalign{\hidewidth$\big|$\hidewidth\cr$\nmid$\cr}%
  }%
}
\newcommand{\ord}{\operatorname{\mathrm{ord}}}
\newcommand{\ind}{\operatorname{\mathrm{ind}}}
\newcommand{\legendre}[2]{\left(\frac{#1}{#2}\right)}
\setmainfont{Times New Roman}
\setmainhangulfont{Nanum Myeongjo}
\setmonofont{SF Mono}
\setlength{\parindent}{1em}
\setlength{\parskip}{0pt}
\linespread{1.2}
%\renewcommand{\arraystretch}{1.5}
\setlength{\tabcolsep}{0.5em}%
\newenvironment{Figure}
  {\par\medskip\noindent\minipage{\linewidth}}
  {\endminipage\par\medskip}
\newcommand{\translation}[1]{\textsuperscript{#1}}

\makeatletter
\renewcommand\NAT@citesuper[3]{\ifNAT@swa
\if*#2*\else#2\NAT@spacechar\fi
\unskip\kern\p@\textsuperscript{\NAT@@open#1\if*#3*\else,\NAT@spacechar#3\fi\NAT@@close}%
   \else #1\fi\endgroup}
\makeatother

\let\oldtabular\tabular% Store a copy of \tabular
\let\endoldtabular\endtabular% Store a copy of \endtabular
\renewenvironment{tabular}[2][\arraystretch]
  {\edef\arraystretch{#1}% Update \arraystretch
   \oldtabular{#2}}% \begin{tabular}[<stretch>]{<col spec>}
  {\endoldtabular}% \end{tabular}

\begin{document}

\title{Linear Algebra (0031)\newline\space Lecture Notes}
\author{Yulwon Rhee (202211342)}
\institute{Department of Computer Science and Engineering, Konkuk University}

\maketitle

\section{Week 1}

\section{Week 2}
\subsection{Multiplication with Vectors}
General rule for $m\times n$ matrix $A$ and $n$-dimensional column vector $\mathbf{x}$,
$$A\mathbf{x} = \begin{bmatrix}
    a_{11}&a_{12}&\cdots&a_{1n}\\
    a_{21}&a_{22}&\cdots&a_{2n}\\
    \vdots&\vdots&\ddots&\vdots\\
    a_{m1}&a_{m2}&\cdots&a_{mn}
\end{bmatrix}\begin{bmatrix}
    x_1\\x_2\\\vdots\\x_n
\end{bmatrix}=\begin{bmatrix}
    \sum^n_{j=1} a_{1j}x_j\\
    \sum^n_{j=1} a_{2j}x_j\\
    \vdots\\
    \sum^n_{j=1} a_{mj}x_j\\
\end{bmatrix}$$
Note: $A$ can be multiplied on the right by vector $\mathbf{x}$ only when the number of columns in $A$ is equal to the dimension of $\mathbf{x}$.

\subsection{Multiplication as Linear Combination}
Multiplication can be expressed as linear combination. For $m\times n$ matrix $A$ and $n$-dimensional column vector $\mathbf{x}$,
$$A\mathbf{x}=\begin{bmatrix}
    \mathbf{a}_1 & \mathbf{a}_2 & \cdots & \mathbf{a}_n
\end{bmatrix} \begin{bmatrix}
    x_1\\x_2\\\vdots\\x_n
\end{bmatrix}=x_1\mathbf{a}_1+x_2\mathbf{a}_2+\cdots+x_n\mathbf{a}_n$$
Note: This is very important view of multiplication whice will be useful throughout this course.

\subsection{Matrix Multiplication}
Generalisation for $m\times n$ matrix $A$ and $n\times p$ matrix $B$
\begin{align*}
    AB&=\begin{bmatrix}
        a_{11}&a_{12}&\cdots&a_{1n}\\
        a_{21}&a_{22}&\cdots&a_{2n}\\
        \vdots&\vdots&\ddots&\vdots\\
        a_{m1}&a_{m2}&\cdots&a_{mn}
    \end{bmatrix}\begin{bmatrix}
        b_{11}&b_{12}&\cdots&b_{1p}\\
        b_{21}&b_{22}&\cdots&b_{2p}\\
        \vdots&\vdots&\ddots&\vdots\\
        b_{n1}&b_{n2}&\cdots&b_{np}
    \end{bmatrix}\\
    &=\begin{bmatrix}
        \sum^n_{k=1}a_{1k}b_{k1}&\sum^n_{k=1}a_{1k}b_{k2}&\cdots&\sum^n_{k=1}a_{1k}b_{kp}\\
        \sum^n_{k=1}a_{2k}b_{k1}&\sum^n_{k=1}a_{2k}b_{k2}&\cdots&\sum^n_{k=1}a_{2k}b_{kp}\\
        \vdots&\vdots&\ddots&\vdots\\
        \sum^n_{k=1}a_{mk}b_{k1}&\sum^n_{k=1}a_{mk}b_{k2}&\cdots&\sum^n_{k=1}a_{mk}b_{kp}
    \end{bmatrix} \in \mathbb{R}^{m\times p}
\end{align*}

Alternative representaion:
\begin{align*}
    AB&=\begin{bmatrix}
        \underline{\mathbf{a}}_1\\\underline{\mathbf{a}}_2\\\vdots\\\underline{\mathbf{a}}_m
    \end{bmatrix}\begin{bmatrix}
        \mathbf{b}_1&\mathbf{b}_2&\cdots&\mathbf{b}_p
    \end{bmatrix}=\begin{bmatrix}
        \underline{\mathbf{a}}_1\mathbf{b}_1&\underline{\mathbf{a}}_1\mathbf{b}_2&\cdots&\underline{\mathbf{a}}_1\mathbf{b}_p\\
        \underline{\mathbf{a}}_1\mathbf{b}_2&\underline{\mathbf{a}}_2\mathbf{b}_2&\cdots&\underline{\mathbf{a}}_2\mathbf{b}_p\\
        \vdots&\vdots&\ddots&\vdots\\
        \underline{\mathbf{a}}_m\mathbf{b}_1&\underline{\mathbf{a}}_m\mathbf{b}_2&\cdots&\underline{\mathbf{a}}_m\mathbf{b}_p
    \end{bmatrix}\\
    &=A\begin{bmatrix}
        \mathbf{b}_1&\mathbf{b}_2&\cdots&\mathbf{b}_p
    \end{bmatrix}=\begin{bmatrix}
        A\mathbf{b}_1&A\mathbf{b}_2&\cdots&A\mathbf{b}_p
    \end{bmatrix}
\end{align*}

- $\left( m \times n \right)\left( n \times p \right) = \left( m \times p \right)$

- $AA = A^2, AAA=A^3, \cdots$ (Only if $A$ is square matrix)\\
Note: Consider $m\times n$ matrix $A$ and $l\times p$ matrix $B$. $A$ can be multiplied on the right by $B$ only if $n=l$.

\subsubsection{Properties of Addition and Multiplication}
Commutativity, Associativity, Distributivity.
\begin{table}[]
    \begin{tabular}{l|l}
    Addition          & Multiplication                                                          \\ \hline
    $A+B=B+A$         & $AB \neq BA$                                                            \\
    $(A+B)+C=A+(B+C)$ & $(AB)C=A(BC)$                                                           \\
    $c(A+B)=cA+cB$    & \begin{tabular}[c]{@{}l@{}}$C(A+B)=CA+CB$\\ $(A+B)C=AC+BC$\end{tabular} \\
    \end{tabular}
\end{table}

\subsection{Transposes}
The transpose of $A$, denoted by $A^T$, is a matrix such that $\begin{bmatrix}
    A^T
\end{bmatrix}_{ij} = A_{ji}$.\\
Properties

- $(A+B)^T=A^T+B^T$

- $(AB)^T = B^TA^T$\\
Inner product of vectors $\mathbf{v}\in\mathbb{R}^n$ and $\mathbf{w}\in\mathbb{R}^n$
$$\mathbf{v}\cdot\mathbf{w} = \mathbf{v}^T\mathbf{w}\in\mathbb{R}$$\\
Outer product of vectors $\mathbf{v}\in\mathbb{R}^n$ and $\mathbf{w}\in\mathbb{R}^n$
$$\mathbf{v}\otimes\mathbf{w}=\mathbf{v}\mathbf{w}^T\in\mathbb{R}^{m\times n}$$
$$\begin{bmatrix}
    \mathbf{vw}^T
\end{bmatrix}_{ij}=v_iw_j$$
generalised to Kronecker product.

\subsection{Some Special Matrices}
$$\Re$$
\end{document}