% !TeX program = xelatex
\documentclass[runningheads]{llncs}
\usepackage[paperheight=295mm,paperwidth=210mm]{geometry}
\usepackage{graphicx}
\usepackage{wrapfig}
\usepackage{import}
\usepackage{kotex}
\usepackage[dvipsnames]{xcolor}
\usepackage{fancyvrb} %
\usepackage{listings}
\usepackage{tabularx}
\usepackage{underscore}
\usepackage{multicol}
\usepackage{enumitem}
\usepackage{subcaption}
\usepackage[numbers,square,super]{natbib}
\usepackage{mathptmx} % Times New Roman
\usepackage{amsmath}
\usepackage{amssymb}
\usepackage{framed}
\usepackage{etoolbox}
\usepackage{cancel}
\usepackage{tikz}
\usepackage{parskip}
\usepackage{enumerate}
\usepackage{minted}
\usepackage{inconsolata}
\usepackage{makecell}
\usepackage{slashed}
\usepackage{nicematrix}
\usetikzlibrary{calc, angles, quotes, graphs, positioning, arrows, graphs.standard}

\setcounter{tocdepth}{2}

\colorlet{shadecolor}{gray!30}

\newcommand\enclosebox[2]{%
  \BeforeBeginEnvironment{#1}{\begin{#2}}%
  \AfterEndEnvironment{#1}{\end{#2}}%
}

\enclosebox{theorem}{oframed}
\enclosebox{definition}{leftbar}

\newcommand{\divides}{\bigm|}
\newcommand{\ndivides}{%
  \mathrel{\mkern.5mu % small adjustment
    % superimpose \nmid to \big|
    \ooalign{\hidewidth$\big|$\hidewidth\cr$\nmid$\cr}%
  }%
}
\newcommand{\ord}{\operatorname{\mathrm{ord}}}
\newcommand{\ind}{\operatorname{\mathrm{ind}}}
\newcommand{\legendre}[2]{\left(\frac{#1}{#2}\right)}
\setmainfont{Times New Roman}
\setmainhangulfont{KoPubWorldBatang_Pro}
\setmonofont{SFMono Nerd Font}
% \setlength{\parindent}{1em}
% \setlength{\parskip}{1em}
\linespread{1.2}
\renewcommand{\arraystretch}{1.5}
\setlength{\tabcolsep}{0.5em}%
\newenvironment{Figure}
{\par\medskip\noindent\minipage{\linewidth}}
{\endminipage\par\medskip}
\newcommand{\translation}[1]{\textsuperscript{#1}}

\makeatletter
\renewcommand\NAT@citesuper[3]{\ifNAT@swa
  \if*#2*\else#2\NAT@spacechar\fi
  \unskip\kern\p@\textsuperscript{\NAT@@open#1\if*#3*\else,\NAT@spacechar#3\fi\NAT@@close}%
  \else #1\fi\endgroup}
\makeatother

\let\oldtabular\tabular% Store a copy of \tabular
\let\endoldtabular\endtabular% Store a copy of \endtabular
\renewenvironment{tabular}[2][\arraystretch]
{\edef\arraystretch{#1}% Update \arraystretch
  \oldtabular{#2}}% \begin{tabular}[<stretch>]{<col spec>}
{\endoldtabular}% \end{tabular}

\setminted{linenos, fontsize=\small, breaklines}

\begin{document}

\title{Java Programming (0409)\newline\space Final Project Report}
\author{Department of Computer Science and Engineering, Konkuk University\newline Yulwon Rhee (202211342)}
\institute{담당 교수: 지정희 교수님\newline제출일: 2022년 12월 5일 월요일}

\maketitle
\section{문제 정의 및 분석}
본 프로젝트에서는 Java와 Swing을 이용하여 빙고 게임 데스크탑 애플리케이션을 제작한다.

\subsection{문제 정의}
\textbf{빙고판 생성:}
사용자에게 $N$\textsuperscript{빙고판의 크기}을 입력받은 후, 단어장 파일에서 $N^2$개의 랜덤한 단어를 골라 빙고판을 두 개 생성한다.

\textbf{턴 진행:}
빙고 게임은 유저와 컴퓨터가 번갈아가며 빙고판의 단어를 선택하는 식으로 진행한다.

\textbf{단어 선택:}
빙고판의 단어가 선택된 경우, 선택된 단어와 그 단어의 뜻을 출력한다.

\textbf{단어 체크:}
유저와 컴퓨터의 빙고판에 선택된 단어가 있는지 확인하고, 체크한다.

\textbf{컴퓨터의 단어 선택:}
컴퓨터가 단어를 선택할 때에는 승리를 위한 알고리즘을 설계하여 그에 따라 선택하도록 한다.

\textbf{빙고 체크:}
한 턴이 진행된 후마다 빙고 수를 체크한다. 빙고의 수가 1개 이상 많은 경우 승리하도록 한다.

\textbf{승률 저장:}
사용자와 컴퓨터의 승률을 파일로 저장하여, 다음 실행 시에도 불러오도록 한다.

\textbf{UI 개선:}
Swing을 이용하여 UI를 개선한다.

\newpage
\section{주요 소스코드 설명}
메소드들은 여러 클래스에 나뉘어 존재한다.

\begin{tabularx}{\textwidth}{l|X}
    \hline
    파일 이름                         & 역할                                  \\
    \hline
    \texttt{BingoGame.java}       & 프로그램의 시작점. 상황에 맞는 Frame을 띄워준다.      \\
    \hline
    \texttt{Game.java}            & 통계를 위한 게임 정보를 관리하는 Class.           \\
    \texttt{User.java}            & 플레이어와 컴퓨터의 게임 중 데이터를 관리하는 Class.    \\
    \texttt{Word.java}            & 단어장에서 단어를 불러오고, 이를 관리하는 Class.      \\
    \hline
    \texttt{StatisticUtil.java}   & 통계를 관리하는 Class.                     \\
    \hline
    \texttt{StartFrame.java}      & 게임을 시작하기 전 필요한 정보를 수집하는 Frame.      \\
    \texttt{GameFrame.java}       & 게임 화면 Frame. 게임과 관련된 동작이 여기서 이루어진다. \\
    \texttt{StatisticDialog.java} & 통계를 자세히 보여주는 Dialog.                \\
    \hline
\end{tabularx}

\subsection{\texttt{BingoGame.java}}
\mintinline{java}{public static void main(String[] args)}:
프로그램의 시작점으로써,
시작 시 저장된 통계 파일을 읽어와 \mintinline{java}{StartFrame}을 호출한다.
이후, \mintinline{java}{StartFrame}에서 단어장 파일, $N$의 값과 AI 수준을 받아와 \mintinline{java}{GameFrame}을 호출한다.
게임이 끝나면, 게임 시작 시간, $N$의 값, AI 수준과 승패 여부를 받아와 통계 리스트에 저장하고, 통계 파일에 추가한다.
% \begin{minted}{java}
% public static void main(String[] args) {
%     StatisticHandler.init();

%     while (true) {
%         StartFrame startFrame = new StartFrame("BINGO GAME", StatisticHandler.getStatisticList());

%         ArrayList<Word> wordList = Word.getWordList(startFrame.getSelectedFile());
%         int N = startFrame.getN();
%         int difficulty = startFrame.getDifficulty();

%         User user = new User(wordList, N);
%         User computer = new User(wordList, N);

%         Game game = new Game(N, difficulty);
%         GameFrame gameFrame = new GameFrame("BINGO GAME", user, computer, N, difficulty);
%         game.setWinLoseInfo(gameFrame.getWinLoseInfo());

%         StatisticHandler.addToStatisticList(game);
%         StatisticHandler.writeStatistic(game);

%     }

% }
% \end{minted}
\subsection{\texttt{Game.java}}
승패 여부, $N$의 값, AI 수준 및 게임 플레이 시간 등 게임 데이터를 관리하는 Class이다.
게임에 필요한 상수가 \mintinline{java}{static}으로 선언되어 있고, 통계 파일에서 데이터를 파싱해 \mintinline{java}{Game} 클래스로 바꿔주는 \mintinline{java}{static} 메소드가 포함되어 있다.

\mintinline{java}{public String getCsvInfo()}:
게임 데이터를 CSV\textsuperscript{Comma-Seperated Values} 형태로 변환해 텍스트 형태로 저장하기 쉽게 반환한다.

\mintinline{java}{public Object[] getTableRow(int winLoseInfo)}:
게임 데이터를 \mintinline{java}{JTable}의 Row에 바로 넣을 수 있도록 가공하여 반환한다.

\mintinline{java}{public static Game parseGameInfo(String gameInfoAsCsv)}:
CSV 형태로 변환된 게임 데이터를 파싱하여 다시 \mintinline{java}{Game} 객체로 되돌려 반환한다.
인스턴스를 생성하지 않아도 바로 쓸 수 있게 \mintinline{java}{static} 메소드로 만들었다.

\newpage
\subsection{\texttt{User.java}}
유저 및 컴퓨터의 빙고판, 빙고 갯수, 단어 목록 등을 묶어 편하게 관리하기 위한 Class이다.

\mintinline{java}{public User(ArrayList<Word> fullWordList, int N)}:
\mintinline{java}{User} 클래스의 생성자로써,\\\mintinline{java}{fullWordList}\textsuperscript{단어장에서 받아온 전체 단어 리스트}와 $N$을 파라미터로 받아 $N^2$ 크기의 빙고판을 만든다.
\mintinline{java}{Collections.shuffle()} 메소드를 이용하여 \mintinline{java}{fullWordList}를 랜덤하게 섞어 맨 앞 $N^2$개의 단어를 선택하는 방식으로 해당 기능을 구현했다.

\mintinline{java}{public void constructBingoPanel()}:
Swing의 \mintinline{java}{GridLayout}을 사용하여 빙고판 레이아웃을 만든다. 빙고판을 다시 그릴 때마다 호출된다.

\mintinline{java}{public boolean isSelectable()}:
플레이어의 빙고판에 선택할 수 있는 단어가 남았는지 여부를 반환한다.

\mintinline{java}{public void updateBingoCount()}:
플레이어의 빙고판에서 빙고의 갯수를 계산한다.

\mintinline{java}{public Word selectWordToWin(int difficulty, ArrayList<Word> comparableList)}:
컴퓨터가 사용하는 메소드로써, 이기기 위해 선택해야 할 단어를 반환한다.
만약 AI 수준이 쉬움일 경우, 빙고판에서 선택되지 않은 단어 중 랜덤으로 하나를 뽑아 반환한다.
AI 수준이 보통일 경우, 가로, 세로, 대각선 중 가장 빙고에 가까운 줄에서 선택되지 않은 단어를 골라 반환한다.
AI 수준이 어려움일 경우, 만약 $N$이 홀수이고 가장 처음 선택되는 단어일 경우 빙고판 한가운데를 먼저 선택해 반환한다.
$N$이 홀수가 아니거나 가장 처음 선택하는 단어가 아닐 경우, 가로, 세로, 대각선 중 가장 빙고에 가까운 줄에서 선택되지 않은 단어를 반환하되,
$1/2$의 확률로 사용자의 빙고판에 없는 단어를 우선적으로 반환하도록 한다.
사용자의 빙고판에 있더라도 그 단어만 선택하면 빙고가 완성되는 경우, 그 단어를 반환하도록 한다.

\subsection{\texttt{Word.java}}
단어의 영어, 한국어 뜻, 빙고판에서 선택 여부, 이 단어를 선택했을 경우 빙고가 될 확률을 묶어 편하게 관리하기 위한 Class이다.
단어장 파일에서 단어를 읽어오는 \mintinline{java}{static} 메소드가 포함되어 있다.

\mintinline{java}{public static ArrayList<Word> getWordList(File file)}:
단어장 파일에서 영단어와 한국어 뜻을 읽어와 \mintinline{java}{Word} 클래스 형태로 바꾸어 리스트에 넣어 반환한다.
인스턴스를 생성하지 않아도 바로 쓸 수 있게 \mintinline{java}{static} 메소드로 만들었다.

\subsection{\texttt{StatisticUtil.java}}
통계를 편리하게 관리하기 위한 유틸리티 Class이다.
인스턴스를 생성하지 않아도 바로 쓸 수 있게 모두 \mintinline{java}{static} 메소드로 만들었다.

\mintinline{java}{public static void init()}:
CSV 형식으로 쓰여진 통계 파일에서 통계를 파싱해 \mintinline{java}{statisticList}에 저장하고,
통계 정보를 기록할 수 있는 \mintinline{java}{FileWriter}를 초기화한다.

\mintinline{java}{public static void addToStatisticList(Game game)}:
게임 데이터를 \mintinline{java}{statisticList}에 저장한다.

\mintinline{java}{public static void writeStatistic(Game game)}:
CSV 형식으로 변환된 게임 데이터를 \mintinline{java}{FileWriter}로 이어쓴 후, 버퍼를 Flush한다.

\mintinline{java}{public static void closeStatisticWriterStream()}:
\mintinline{java}{FileWriter}의 \mintinline{java}{InputStream}을 \mintinline{java}{close}한다.

\subsection{\texttt{StartFrame.java}}
게임을 시작하기 전 필요한 정보인 단어장 파일, AI 수준, $N$을 입력받고, 저장된 통계 정보를 출력하는 Frame이다.

\mintinline{java}{public StartFrame(String title, ArrayList<Game> statisticList)}:
\mintinline{java}{StartFrame}의 생성자로써, 윈도우의 타이틀과 통계 목록을 파라미터로 받아 클래스 내에서 사용 가능하도록 한다.
게임을 시작하기 위해 필요한 정보가 입력될 때까지 메인 스레드에서 입력 받은 값을 접근할 수 없도록 \mintinline{java}{this.wait()}을 사용한다.

\mintinline{java}{private void initUI()}:
게임에 필요한 정보를 수집할 UI를 구성한다.
생성자를 통해 넘겨받은 통계 정보를 이용해 화면에 승률을 표시하고, 클릭 시 \mintinline{java}{StatisticDialog}가 호출되도록 한다.

\mintinline{java}{public void actionPerformed(ActionEvent e)}:
\mintinline{java}{ActionListener}의 \mintinline{java}{actionPerformed}를 Override한 메소드로,
시작하기 버튼이 눌렸거나 $N$을 입력받는 텍스트 필드에서 Enter 키가 눌렸을 때 호출된다.
선택된 단어장 파일의 존재 여부와, $N$값의 유효성(단어장의 단어 갯수가 $N^2$ 보다 큰지, $3 \leq N \leq 10$인지)을 체크하여 정상적인 실행이 불가할 경우 경고 Dialog를 띄운다.
만약 정상적으로 게임이 실행 가능하면 메인 스레드에서 \mintinline{java}{GameFrame}을 호출할 수 있도록 \mintinline{java}{this.notify()}해주고 윈도우를 \mintinline{java}{dispose()}한다.

\mintinline{java}{private int countLine(File file)}:
파라미터로 전달받은 단어장 파일의 단어 갯수를 반환한다.

\subsection{\texttt{GameFrame.java}}
게임 화면을 출력하는 Frame이다.

\mintinline{java}{public GameFrame(String title, User user, User computer, int N, int difficulty)}:
\mintinline{java}{GameFrame}의 생성자로써, 윈도우 타이틀, 유저 및 컴퓨터의 플레이 데이터, $N$값과 AI 수준을 파라미터로 입력받아 Class 내에서 사용 가능하도록 한다.
게임 데이터가 완성될 때까지 메인 스레드에서 입력받은 값을 접근할 수 없도록 \mintinline{java}{this.wait()}을 사용한다.

\mintinline{java}{private void initUI()}:
게임 플레이 UI를 구성한다.

\mintinline{java}{public void actionPerformed(ActionEvent e)}:
\mintinline{java}{ActionListener}의 \mintinline{java}{actionPerformed}를 Override한 메소드로,
단어 입력 버튼이 눌렸거나 단어를 입력받는 텍스트 필드에서 Enter키가 눌렸을 때 호출된다.



\subsection{\texttt{StatisticDialog.java}}


\section{실행 결과}
\section{느낀점 및 토의 사항}
\section{기타}
\subsection{개발 환경}
이 프로젝트를 개발하는 과정에서, 운영체제는 macOS 13.1 Ventura를 사용하였고, IDE는 Jetbrain사의 IntelliJ IDEA 2022.2.3 버전을 사용하였다.\\
JDK는 Temurin의 OpenJDK를 사용하였고, 버전 정보는 다음과 같다.\\
\texttt{
    openjdk 17.0.5 2022-10-18\\
    OpenJDK Runtime Environment Temurin-17.0.5+8 (build 17.0.5+8)\\
    OpenJDK 64-Bit Server VM Temurin-17.0.5+8 (build 17.0.5+8, mixed mode, sharing)
}

\subsection{수행 기간}
2022.11.28(월) - 2022.11.30(수)
\end{document}
