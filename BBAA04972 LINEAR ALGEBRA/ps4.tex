% !TeX program = xelatex
\documentclass[runningheads]{llncs}
\usepackage[paperheight=295mm,paperwidth=210mm]{geometry}
\usepackage{graphicx}
\usepackage{wrapfig}
\usepackage{import}
\usepackage{kotex}
\usepackage[dvipsnames]{xcolor}
\usepackage{fancyvrb} %
\usepackage{listings}
\usepackage{tabularx}
\usepackage{underscore}
\usepackage{multicol}
\usepackage{enumitem}
\usepackage{subcaption}
\usepackage[numbers,square,super]{natbib}
\usepackage{mathptmx} % Times New Roman
\usepackage{amsmath}
\usepackage{amssymb}
\usepackage{framed}
\usepackage{etoolbox}
\usepackage{cancel}
\usepackage{physics}
\usepackage{tikz}
\usepackage{parskip}
\usepackage{enumerate}
\usepackage{minted}
\usepackage{inconsolata}
\usepackage{makecell}
\usepackage{slashed}
\usepackage{nicematrix}
\usetikzlibrary{calc, angles, quotes, graphs, positioning, arrows}

\setcounter{tocdepth}{2}

\colorlet{shadecolor}{gray!30}

\newcommand\enclosebox[2]{%
  \BeforeBeginEnvironment{#1}{\begin{#2}}%
  \AfterEndEnvironment{#1}{\end{#2}}%
}

\enclosebox{theorem}{oframed}
\enclosebox{definition}{leftbar}

\newcommand{\divides}{\bigm|}
\newcommand{\ndivides}{%
  \mathrel{\mkern.5mu % small adjustment
    % superimpose \nmid to \big|
    \ooalign{\hidewidth$\big|$\hidewidth\cr$\nmid$\cr}%
  }%
}
\newcommand{\ord}{\operatorname{\mathrm{ord}}}
\newcommand{\ind}{\operatorname{\mathrm{ind}}}
\newcommand{\legendre}[2]{\left(\frac{#1}{#2}\right)}
\setmainfont{Times New Roman}
\setmainhangulfont{Nanum Myeongjo}
\setmonofont{SF Mono}
\setlength{\parindent}{1em}
\setlength{\parskip}{0pt}
\linespread{1.2}
%\renewcommand{\arraystretch}{1.5}
\setlength{\tabcolsep}{0.5em}%
\newenvironment{Figure}
  {\par\medskip\noindent\minipage{\linewidth}}
  {\endminipage\par\medskip}
\newcommand{\translation}[1]{\textsuperscript{#1}}

\makeatletter
\renewcommand\NAT@citesuper[3]{\ifNAT@swa
\if*#2*\else#2\NAT@spacechar\fi
\unskip\kern\p@\textsuperscript{\NAT@@open#1\if*#3*\else,\NAT@spacechar#3\fi\NAT@@close}%
   \else #1\fi\endgroup}
\makeatother

\let\oldtabular\tabular% Store a copy of \tabular
\let\endoldtabular\endtabular% Store a copy of \endtabular
\renewenvironment{tabular}[2][\arraystretch]
  {\edef\arraystretch{#1}% Update \arraystretch
   \oldtabular{#2}}% \begin{tabular}[<stretch>]{<col spec>}
  {\endoldtabular}% \end{tabular}

\begin{document}

\title{Linear Algebra (0031)\newline\space Problem Set 4 Solutions}
\author{Yulwon Rhee (202211342)}
\institute{Department of Computer Science and Engineering, Konkuk University}

\maketitle
\subsubsection{1.}
(a) Compute the ranks of $A,\ B$.
\paragraph*{Solution.}
$$A=\begin{bmatrix}
    1&1&6\\
    1&-1&2\\
    2&-1&6\\
    1&2&8
\end{bmatrix}\Rightarrow\begin{bmatrix}
    1&0&4\\
    0&1&2\\
    0&0&0\\
    0&0&0
\end{bmatrix}$$
$$\therefore r(A) = 2$$\\
$$B=\begin{bmatrix}
    1&2&3\\
    1&2&3\\
    1&2&3\\
    1&2&3
\end{bmatrix}\Rightarrow\begin{bmatrix}
    1&2&3\\
    0&0&0\\
    0&0&0\\
    0&0&0
\end{bmatrix}$$
$$\therefore r(B) = 1$$\\

(b) Find the bases of $C(A),\ N(A^T),\ C(A^T),\ N(A)$.
\paragraph*{Solution.}
$$A=\begin{bmatrix}
    1&1&6\\
    1&-1&2\\
    2&-1&6\\
    1&2&8
\end{bmatrix}\Rightarrow\begin{bmatrix}
    1&0&4\\
    0&1&2\\
    0&0&0\\
    0&0&0
\end{bmatrix}$$
$$\therefore C(A)=\left\{\begin{bmatrix}
    1\\1\\2\\1
\end{bmatrix}, \begin{bmatrix}
    1\\-1\\-1\\2
\end{bmatrix}\right\}$$
We can get the basis by just using the pivot columns from the original matrix.\\

$$A^T=\begin{bmatrix}
    1&1&2&1\\1&-1&-1&2\\6&2&6&8
\end{bmatrix}\Rightarrow\begin{bmatrix}
    1&0&\frac{1}{2}&\frac{3}{2}\\
    0&1&\frac{3}{2}&-\frac{1}{2}\\
    0&0&0&0
\end{bmatrix}$$
$$\begin{bmatrix}
1&0&\frac{1}{2}&\frac{3}{2}\\
0&1&\frac{3}{2}&-\frac{1}{2}\\
0&0&0&0
\end{bmatrix}\begin{bmatrix}
    x_1\\x_2\\x_3\\x_4
\end{bmatrix}=\begin{bmatrix}
    0\\0\\0
\end{bmatrix}$$
Let $x_1=-\frac{3}{2}x_4-\frac{1}{2}x_3, x_2=\frac{1}{2}x_4-\frac{3}{2}x_3$.
$$\mathbf{x}=\begin{bmatrix}
    -\frac{3}{2}x_4-\frac{1}{2}x_3\\
    \frac{1}{2}x_4-\frac{3}{2}x_3\\
    x_3\\
    x_4
\end{bmatrix}=\begin{bmatrix}
    -\frac{1}{2}\\
    -\frac{3}{2}\\
    1\\
    0
\end{bmatrix}x_3+\begin{bmatrix}
    -\frac{3}{2}\\
    \frac{1}{2}\\
    0\\
    1
\end{bmatrix}x_4$$
$$\therefore N(A^T)=\left\{\begin{bmatrix}
    -\frac{1}{2}\\
    -\frac{3}{2}\\
    1\\
    0
\end{bmatrix},\begin{bmatrix}
    -\frac{3}{2}\\
    \frac{1}{2}\\
    0\\
    1
\end{bmatrix}\right\}$$\\

$$A^T=\begin{bmatrix}
    1&1&2&1\\1&-1&-1&2\\6&2&6&8
\end{bmatrix}\Rightarrow\begin{bmatrix}
    1&0&\frac{1}{2}&\frac{3}{2}\\
    0&1&\frac{3}{2}&-\frac{1}{2}\\
    0&0&0&0
\end{bmatrix}$$
$$\therefore C(A^T) = \left\{\begin{bmatrix}
    1\\1\\6
\end{bmatrix},\begin{bmatrix}
    1\\-1\\2
\end{bmatrix}\right\}$$\\

$$A=\begin{bmatrix}
    1&1&6\\
    1&-1&2\\
    2&-1&6\\
    1&2&8
\end{bmatrix}\Rightarrow\begin{bmatrix}
    1&0&4\\
    0&1&2\\
    0&0&0\\
    0&0&0
\end{bmatrix}$$
$$\begin{bmatrix}
    1&0&4\\
    0&1&2\\
    0&0&0\\
    0&0&0
\end{bmatrix}\begin{bmatrix}
    x_1\\x_2\\x_3
\end{bmatrix}=\begin{bmatrix}
    0\\0\\0\\0
\end{bmatrix}$$
$$\mathbf{x}=\begin{bmatrix}
    -4x_3\\
    -2x_3\\
    x_3
\end{bmatrix}=\begin{bmatrix}
    -4\\-2\\1
\end{bmatrix}x_3$$
$$\therefore N(A)=\left\{\begin{bmatrix}
    -4\\-2\\1
\end{bmatrix}\right\}$$\\

(c) Find the dimensions of $C(A),\ N(A^T),\ C(A^T),\ N(A)$.
\begin{gather*}
    dim(C(A)) = 2\\dim(N(A^T))=2\\dim(C(A^T))=2\\dim(N(A)) = 1
\end{gather*}\\

(d) What is the rank of $\begin{bmatrix}
    A\\A
\end{bmatrix}$? Compare with the rank of A.
Prove or disprove if your result is true in general.

$$\begin{bmatrix}
    A\\A
\end{bmatrix}=\begin{bmatrix}
    1&1&6\\
    1&-1&2\\
    2&-1&6\\
    1&2&8\\
    1&1&6\\
    1&-1&2\\
    2&-1&6\\
    1&2&8
\end{bmatrix}\Rightarrow\begin{bmatrix}
    1&0&4\\
    0&1&2\\
    0&0&0\\
    0&0&0\\
    0&0&0\\
    0&0&0\\
    0&0&0\\
    0&0&0
\end{bmatrix}$$
$$\therefore r\left(\begin{bmatrix}
    A\\A
\end{bmatrix}\right) = 2 = r(A)$$
Since value of the rank is equals to the number of pivots, It will be true for any matrix A.\\

(e) What is the rank of $\begin{bmatrix}
    B\\A^TB
\end{bmatrix}$? Compare with the rank of B.
Prove or disprove if your result is true in general.

\begin{align*}
    A^TB&=\begin{bmatrix}
        1&1&2&1\\
        1&-1&-1&2\\
        6&2&6&8
    \end{bmatrix}\begin{bmatrix}
        1&2&3\\
        1&2&3\\
        1&2&3\\
        1&2&3
    \end{bmatrix}\\
    &=\begin{bmatrix}
        5&10&15\\
        3&6&9\\
        22&44&66
    \end{bmatrix}
\end{align*}
$$\begin{bmatrix}
    B\\
    A^TB
\end{bmatrix}=\begin{bmatrix}
    1&2&3\\
    1&2&3\\
    1&2&3\\
    1&2&3\\
    5&10&15\\
    3&6&9\\
    22&44&66
\end{bmatrix}\Rightarrow\begin{bmatrix}
    1&2&3\\
    0&0&0\\
    0&0&0\\
    0&0&0\\
    0&0&0\\
    0&0&0\\
    0&0&0
\end{bmatrix}$$
$$\therefore r\left(\begin{bmatrix}
    B\\A^TB
\end{bmatrix}\right)=1=r(B)$$
ASDFASDFASDFASDF\\

(f) What is the rank of $\begin{bmatrix}
    A & B
\end{bmatrix}$?
$$\begin{bmatrix}
    A&B
\end{bmatrix}=\begin{bmatrix}
    1&1&6&1&2&3\\
    1&-1&2&1&2&3\\
    2&-1&6&1&2&3\\
    1&2&8&1&2&3
\end{bmatrix}\Rightarrow\begin{bmatrix}
    1&0&4&0&0&0\\
    0&1&2&0&0&0\\
    0&0&0&1&2&3\\
    0&0&0&0&0&0
\end{bmatrix}$$
$$r\left(\begin{bmatrix}
    A&B
\end{bmatrix}\right)=3$$\\

(g) Let $C=\begin{bmatrix}
    1&5&3\\
    1&1&3\\1&4&3\\1&7&3
\end{bmatrix}$. What is the rank of $\begin{bmatrix}
    A&C
\end{bmatrix}$? Compare the results in parts (f) and (g), and explain why.
$$\begin{bmatrix}
    A&C
\end{bmatrix}=\begin{bmatrix}
    1&1&6&1&5&3\\
    1&-1&2&1&1&3\\
    2&-1&6&1&4&3\\
    1&2&8&1&7&3
\end{bmatrix}\Rightarrow\begin{bmatrix}
    1&0&4&0&3&0\\
    0&1&2&0&2&0\\
    0&0&0&1&0&3\\
    0&0&0&0&0&0
\end{bmatrix}$$
$$\therefore r\left(\begin{bmatrix}
    A&C
\end{bmatrix}\right)=3$$\\

(h) Compute the rank of $A+B$. Is it greater than $r(A) + r(B)$?
$$A+B=\begin{bmatrix}
    2&3&9\\2&1&5\\3&1&9\\2&4&11
\end{bmatrix}\Rightarrow\begin{bmatrix}
    1&0&0\\
    0&1&0\\
    0&0&1\\
    0&0&0
\end{bmatrix}$$
$$\therefore r(A+B) = 3$$
$$\therefore r(A+B) = r(A)+r(B)$$\\

(i) Compute the rank of $A+C$. Is it greater than $r(A) + r(C)$?
$$A+C=\begin{bmatrix}
    2&6&9\\2&0&5\\3&3&9\\2&9&11
\end{bmatrix}\Rightarrow\begin{bmatrix}
    1&0&0\\
    0&1&0\\
    0&0&1\\
    0&0&0
\end{bmatrix}$$
$$\therefore r(A+C) = 3$$
$$\therefore r(A+C) < r(A)+r(C) (\because r(C) = 2)$$\\

\subsubsection*{2.}
(a) Find the bases of $S_1$ and $S_2$.
$$S_1=\left\{\begin{bmatrix}
    a\\2\\1\\1
\end{bmatrix}, \begin{bmatrix}
    2\\b\\2\\3
\end{bmatrix}\right\}$$
$$S_2=\left\{\begin{bmatrix}
    1\\1\\c\\1
\end{bmatrix},\begin{bmatrix}
    1\\-1\\1\\d
\end{bmatrix}\right\}$$\\

(b) Determine the values of $a, b, c, d$ so that $S_1$ and $S_2$ are orthogonal complement of each other.
$$\begin{bmatrix}
        a&2&1&1\\
        2&b&2&3
    \end{bmatrix}\begin{bmatrix}
        1&1\\
        1&-1\\
        c&1\\
        1&d
    \end{bmatrix}=\begin{bmatrix}
        a+c+3&a+d-1\\
        b+2c+5&-b+3d+4
    \end{bmatrix}=\begin{bmatrix}
        0&0\\
        0&0
    \end{bmatrix}
$$
$$\therefore a=\frac{6}{5},\quad b=\frac{17}{5},\quad c=-\frac{21}{5},\quad d=-\frac{1}{5}$$
\end{document}