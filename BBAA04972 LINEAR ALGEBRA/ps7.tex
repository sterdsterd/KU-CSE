% !TeX program = xelatex
\documentclass[runningheads]{llncs}
\usepackage[paperheight=295mm,paperwidth=210mm]{geometry}
\usepackage{graphicx}
\usepackage{wrapfig}
\usepackage{import}
\usepackage{kotex}
\usepackage[dvipsnames]{xcolor}
\usepackage{fancyvrb} %
\usepackage{listings}
\usepackage{tabularx}
\usepackage{underscore}
\usepackage{multicol}
\usepackage{enumitem}
\usepackage{subcaption}
\usepackage[numbers,square,super]{natbib}
\usepackage{mathptmx} % Times New Roman
\usepackage{amsmath}
\usepackage{amssymb}
\usepackage{framed}
\usepackage{etoolbox}
\usepackage{cancel}
\usepackage{tikz}
\usepackage{parskip}
\usepackage{enumerate}
\usepackage{minted}
\usepackage{inconsolata}
\usepackage{makecell}
\usepackage{slashed}
\usepackage{nicematrix}
\usetikzlibrary{calc, angles, quotes, graphs, positioning, arrows, graphs.standard}

\setcounter{tocdepth}{2}

\colorlet{shadecolor}{gray!30}

\newcommand\enclosebox[2]{%
  \BeforeBeginEnvironment{#1}{\begin{#2}}%
  \AfterEndEnvironment{#1}{\end{#2}}%
}

\enclosebox{theorem}{oframed}
\enclosebox{definition}{leftbar}

\newcommand{\divides}{\bigm|}
\newcommand{\ndivides}{%
  \mathrel{\mkern.5mu % small adjustment
    % superimpose \nmid to \big|
    \ooalign{\hidewidth$\big|$\hidewidth\cr$\nmid$\cr}%
  }%
}
\newcommand{\ord}{\operatorname{\mathrm{ord}}}
\newcommand{\ind}{\operatorname{\mathrm{ind}}}
\newcommand{\legendre}[2]{\left(\frac{#1}{#2}\right)}
\setmainfont{Times New Roman}
\setmainhangulfont{KoPubWorldBatang_Pro}
\setmonofont{SFMono Nerd Font}
% \setlength{\parindent}{1em}
% \setlength{\parskip}{1em}
\linespread{1.2}
\renewcommand{\arraystretch}{1.5}
\setlength{\tabcolsep}{0.5em}%
\newenvironment{Figure}
{\par\medskip\noindent\minipage{\linewidth}}
{\endminipage\par\medskip}
\newcommand{\translation}[1]{\textsuperscript{#1}}

\makeatletter
\renewcommand\NAT@citesuper[3]{\ifNAT@swa
  \if*#2*\else#2\NAT@spacechar\fi
  \unskip\kern\p@\textsuperscript{\NAT@@open#1\if*#3*\else,\NAT@spacechar#3\fi\NAT@@close}%
  \else #1\fi\endgroup}
\makeatother

\let\oldtabular\tabular% Store a copy of \tabular
\let\endoldtabular\endtabular% Store a copy of \endtabular
\renewenvironment{tabular}[2][\arraystretch]
{\edef\arraystretch{#1}% Update \arraystretch
  \oldtabular{#2}}% \begin{tabular}[<stretch>]{<col spec>}
{\endoldtabular}% \end{tabular}

\setminted{linenos, fontsize=\small, breaklines}

\begin{document}

\title{Linear Algebra (0031)\newline\space Problem Set 7 Solutions}
\author{Yulwon Rhee (202211342)}
\institute{Department of Computer Science and Engineering, Konkuk University}

\maketitle
\subsubsection{1.} Consider the Matrix $A = \begin{bmatrix}
    1&0&1\\
    -1&2&1\\
    -2&0&4
\end{bmatrix}$

(a) Calculate the eigenvalues and eigenvectors of $A$.
\paragraph*{Solution.}
\begin{align*}
    |A-\lambda I| &= \begin{vmatrix}
        1-\lambda&0&1\\
        -1&2-\lambda&1\\
        -2&0&4-\lambda
    \end{vmatrix}\\
    &= -\lambda^3 + 7\lambda^2-16\lambda+12\\
    &= -(\lambda - 3)(\lambda - 2)^2\\
    &= 0
\end{align*}
$$\therefore \lambda_1 = 3,\ \lambda_2 = 2$$\\
For every $\lambda$, we find its own vectors:

1. $\lambda_1 = 3$

$$A - \lambda_1 I = \begin{bmatrix}
    -2&0&1\\-1&-1&1\\-2&0&1
\end{bmatrix} \Rightarrow \mathbf{v}_1=\begin{bmatrix}
    \dfrac{1}{2}\\[0.3cm]
    \dfrac{1}{2}\\[0.3cm]
    1
\end{bmatrix}$$

2. $\lambda_1 = 2$

$$A - \lambda_1 I = \begin{bmatrix}
    -1&0&1\\-1&0&1\\-2&0&2
\end{bmatrix} \Rightarrow \mathbf{v}_2=\begin{bmatrix}
    0\\1\\0
\end{bmatrix},\ \mathbf{v}_3=\begin{bmatrix}
    1\\0\\1
\end{bmatrix}$$
\newpage
(b) If $A$ has three linearly independent eigenvectors, find the diagonaliation of $A$
\paragraph{Solution.}
$$P=\begin{bmatrix}
    \dfrac{1}{2} & 0 & 1\\[0.3cm]
    \dfrac{1}{2} & 1 & 0\\[0.3cm]
    1 & 0 & 1
\end{bmatrix},\quad D=\begin{bmatrix}
    3 & 0 & 0\\
    0 & 2 & 0\\
    0 & 0 & 2
\end{bmatrix}$$
$$PDP^{-1} = \begin{bmatrix}
    1&0&1\\
    -1&2&1\\
    -2&0&4
\end{bmatrix}$$

(c) What are the eigenvalues and eigenvectors of $A^{-1}$ (if $A^{-1}$ exists).
\paragraph{Solution.}
$$A^{-1} = \begin{bmatrix}
    \dfrac{2}{3} & 0 & -\dfrac{1}{6}\\[0.3cm]
    \dfrac{1}{6} & 0 & -\dfrac{1}{6}\\[0.3cm]
    \dfrac{1}{3} & 0 & \dfrac{1}{6}
\end{bmatrix}$$
$$|A^T-\lambda I| = -\lambda^3 + \frac{5}{6}\lambda^2-\frac{1}{6}\lambda$$
$$\lambda_1 = 0,\quad\lambda_2 = \frac{1}{3},\quad\lambda_3 = \frac{1}{2}$$

1. $\lambda_1 = 0$

$$A - \lambda_1 I = \begin{bmatrix}
    2/3&0&-1/6\\
    1/6&0&-1/6\\
    1/3&0&1/6
\end{bmatrix} \Rightarrow \mathbf{v}_1=\begin{bmatrix}
    0\\1\\0
\end{bmatrix}$$

2. $\lambda_2 = \dfrac{1}{3}$

$$A - \lambda_2 I = \begin{bmatrix}
    1/3&0&-1/6\\
    1/6&-1/3&-1/6\\
    1/3&0&1/6
\end{bmatrix} \Rightarrow \mathbf{v}_2=\begin{bmatrix}
    1/2\\-1/4\\1
\end{bmatrix}$$

3. $\lambda_3 = \dfrac{1}{2}$

$$A - \lambda_3 I = \begin{bmatrix}
    1/6&0&-1/6\\
    1/6&-1/2&-1/6\\
    1/3&0&-1/3
\end{bmatrix} \Rightarrow \mathbf{v}_3=\begin{bmatrix}
    1\\0\\1
\end{bmatrix}$$
\newpage
(d) Let $B = A + 3I$, where $I$ is the identity matrix. Find the eigenvalues of $B$.
\paragraph{Solution.}
$$B = \begin{bmatrix}
    4&0&1\\
    -1&5&1\\
    -2&0&7
\end{bmatrix}$$
\begin{align*}
    \begin{vmatrix}
        4-\lambda&0&1\\
        -1&5-\lambda&1\\
        -2&0&7-\lambda
    \end{vmatrix} &= -\lambda^3 + 16\lambda^2 - 85\lambda + 150\\
    &= -(\lambda - 6)(\lambda - 5)^2 = 0
\end{align*}
$$\therefore \lambda_1 = 5,\quad \lambda_2 = 6$$

\subsubsection{2.} Consider a square matrix $A$. Suppose that $A$ has full column rank. Can $A$ have eigenvalue $0$? Justify your answer.
\paragraph{Solution.}\phantom{}\\
If $A$ is a square matrix, the $r(A) + \mathrm{dim}(N(A)) = n$; which is the rank-nullity theorem.
The nullity is the dimension of the null space of the matrix, which is all vectors $\mathbf{v}$ of the form:
$$A\mathbf{v} = 0 = 0\mathbf{v}$$
The null space of $A$ is precisely the eigenspace corresponding to eigenvalue $0$.
\\\\\\
\subsubsection{3.} The characteristic polynomial equation (CPE) of $A$ is written as $|A - \lambda I| = 0$. Likewise, the characteristic polynomial equation of $AT$ is $|AT - \lambda I| = 0$. Solving CPE gives eigenvalues.

(a) Show that $A$ and $A^T$ have the same eigenvalues (Hint. Take a look at the CPEs of $A$ and $A^T$, and use the fact that transposing a matrix does not change the determinant)
\paragraph{Solution.}\phantom{}\\
The matrix $(A-\lambda I)^T$ is equals to the matrix $(A^T-\lambda I)$, since the matrix $I$ is symmetric.
Thus, $$|A^T - \lambda I| = |(A - \lambda I)^T| = |A-\lambda I|$$
From above, it is obvious that the eigenvalues are the same for both $A$ and $A^T$.
\newpage
\subsubsection{4.}
Consider a matrix $A = \begin{bmatrix}2&a\\1&0\end{bmatrix}$. Find a condition for $A$ to be diagonalisable.
(Hint. $A$ needs to have two linearly independent eigenvectors. Eigenvectors associated to distinct eigenvalues are linearly independent)
\paragraph{Solution.}\phantom{}\\
Matrix $A$ is diagonalisable iff the sum of the dimension of eigenspaces is equal to $n$.
\begin{align*}
    |A-\lambda I| &= \begin{vmatrix}
        2-\lambda&a\\
        1&-\lambda
    \end{vmatrix}\\
    &= \lambda^2 - 2\lambda + a
\end{align*}
$$\therefore a \neq 1$$
\subsubsection{5.} Check if the following matrices are positive definite:

(a) $\begin{bmatrix}
    2&2&0\\
    2&5&3\\
    0&3&8
\end{bmatrix}$
\paragraph{Solution.}
$$|2| = 2$$
$$\begin{vmatrix}
    2&2\\
    2&5
\end{vmatrix}=6$$
$$\begin{vmatrix}
    2&2&0\\
    2&5&3\\
    0&3&8
\end{vmatrix}=30$$
Since all determinants are positive, the matrix (a) is a positive definite.


(b) $\begin{bmatrix}
    1&2\\3&4
\end{bmatrix}$
\paragraph{Solution.}
\begin{align*}
    |A-\lambda I| &= \begin{vmatrix}
        1-\lambda & 2\\3 & 4 - \lambda
    \end{vmatrix}\\
    &= \lambda^2 - 5\lambda - 2\\
    &= (\lambda + \frac{\sqrt{33} - 5}{2})(\lambda - \frac{\sqrt{33} + 5}{2})
\end{align*}
$$\therefore \lambda_1 = \frac{-\sqrt{33}+5}{2},\quad \lambda_2 = \frac{\sqrt{33}+5}{2}$$
Since all eigenvalues are not positive, the matrix (b) is not a positive definite.
\newpage
\subsubsection{6.} Show that $R^TR$ is positive semidefinite for any matrix $R$.
\paragraph{Solution.}\phantom{}\\
Let $R = \begin{bmatrix}
    1&2\\3&4
\end{bmatrix}$.
$$R^TR = \begin{bmatrix}
    1 &3\\2&4
\end{bmatrix}\begin{bmatrix}
    1&2\\3&4
\end{bmatrix} = \begin{bmatrix}
    5 & 11\\11 & 25
\end{bmatrix}$$
$$|R^TR-\lambda I| = \begin{vmatrix}
    5 - \lambda & 11\\11 & 25 - \lambda
\end{vmatrix}$$
$$\therefore \lambda_1 = 15 + \sqrt{221},\ \lambda_2 = 15 - \sqrt{221}$$
Since all eigenvalues are positive, the matrix $R^TR$ is positive semidefinite.

\subsubsection{7.} Show that $R^TR$ is positive definite if and only if $R$ has full column rank.
\paragraph{Solution.}\phantom{}\\
Let $R = \begin{bmatrix}
    1&1\\1&2
\end{bmatrix}$.
$$R^TR = \begin{bmatrix}
    1 &1\\1&2
\end{bmatrix}\begin{bmatrix}
    1&1\\1&2
\end{bmatrix} = \begin{bmatrix}
    2&3\\3&5
\end{bmatrix}$$
$$|R^TR-\lambda I| = \begin{vmatrix}
    2 - \lambda & 3\\3 & 5 - \lambda
\end{vmatrix}$$
$$\therefore \lambda_1 = \frac{7+3\sqrt{5}}{2},\ \lambda_2 = \frac{7-3\sqrt{5}}{2}$$
Since all eigenvalues are positive, the matrix $R^TR$ is positive definite.
\newpage
\subsubsection{8.} Prove that if $B = M^{-1}AM$, then $A$ and $B$ have the same eigenvalues. (Hint: multiply an eigenvector of $B$ on the right, and then multiply $M$ on the left)
\paragraph{Solution.}\phantom{}\\
Let $X_i$ be an eigenvector of $A$ corresponding to $\lambda_i$, and let the eigenvectors be independent of each other.
$$AX_i=\lambda_i X_i$$
Let $M=[X_1\ X_2\ \cdots\ X_n]$ which contains the columns with the eigenvectors of $A$.\\
So,
\begin{align*}
    AM &= [AX_1\ AX_2\ \cdots\ AX_n]\\
    &=[\lambda_1X_1\ \lambda_2 X2\ \cdots\ \lambda_nX_n]\\
    &=[X_1\ X_2\ \cdots\ X_n]\begin{bmatrix}
        \lambda_1 & 0 & \cdots & 0\\
        0 & \lambda_2 & \cdots & 0\\
        \vdots & \vdots & \ddots & \vdots\\
        0 & 0 & \cdots & \lambda_n
    \end{bmatrix}
\end{align*}
$$AM=MB$$
$B$ is the diagonal matrix. $M$ is invertible as the eigenvectors are independent and $|M|\neq 0$.\\
Therefore, $$M^{-1}AM=B$$

\end{document}