% !TeX program = xelatex
\documentclass[runningheads]{llncs}
\usepackage[paperheight=295mm,paperwidth=210mm]{geometry}
\usepackage{graphicx}
\usepackage{wrapfig}
\usepackage{import}
\usepackage{kotex}
\usepackage[dvipsnames]{xcolor}
\usepackage{fancyvrb} %
\usepackage{listings}
\usepackage{tabularx}
\usepackage{underscore}
\usepackage{multicol}
\usepackage{enumitem}
\usepackage{subcaption}
\usepackage[numbers,square,super]{natbib}
\usepackage{mathptmx} % Times New Roman
\usepackage{amsmath}
\usepackage{amssymb}
\usepackage{framed}
\usepackage{etoolbox}
\usepackage{cancel}
\usepackage{physics}
\usepackage{tikz}
\usepackage{parskip}
\usepackage{enumerate}
\usepackage{minted}
\usepackage{inconsolata}
\usepackage{makecell}
\usepackage{slashed}
\usepackage{nicematrix}
\usetikzlibrary{calc, angles, quotes, graphs, positioning, arrows}

\setcounter{tocdepth}{2}

\colorlet{shadecolor}{gray!30}

\newcommand\enclosebox[2]{%
  \BeforeBeginEnvironment{#1}{\begin{#2}}%
  \AfterEndEnvironment{#1}{\end{#2}}%
}

\enclosebox{theorem}{oframed}
\enclosebox{definition}{leftbar}

\newcommand{\divides}{\bigm|}
\newcommand{\ndivides}{%
  \mathrel{\mkern.5mu % small adjustment
    % superimpose \nmid to \big|
    \ooalign{\hidewidth$\big|$\hidewidth\cr$\nmid$\cr}%
  }%
}
\newcommand{\ord}{\operatorname{\mathrm{ord}}}
\newcommand{\ind}{\operatorname{\mathrm{ind}}}
\newcommand{\legendre}[2]{\left(\frac{#1}{#2}\right)}
\setmainfont{Times New Roman}
\setmainhangulfont{Nanum Myeongjo}
\setmonofont{SF Mono}
\setlength{\parindent}{1em}
\setlength{\parskip}{0pt}
\linespread{1.2}
%\renewcommand{\arraystretch}{1.5}
\setlength{\tabcolsep}{0.5em}%
\newenvironment{Figure}
  {\par\medskip\noindent\minipage{\linewidth}}
  {\endminipage\par\medskip}
\newcommand{\translation}[1]{\textsuperscript{#1}}

\makeatletter
\renewcommand\NAT@citesuper[3]{\ifNAT@swa
\if*#2*\else#2\NAT@spacechar\fi
\unskip\kern\p@\textsuperscript{\NAT@@open#1\if*#3*\else,\NAT@spacechar#3\fi\NAT@@close}%
   \else #1\fi\endgroup}
\makeatother

\let\oldtabular\tabular% Store a copy of \tabular
\let\endoldtabular\endtabular% Store a copy of \endtabular
\renewenvironment{tabular}[2][\arraystretch]
  {\edef\arraystretch{#1}% Update \arraystretch
   \oldtabular{#2}}% \begin{tabular}[<stretch>]{<col spec>}
  {\endoldtabular}% \end{tabular}

\begin{document}

\title{Linear Algebra (0031)\newline\space Problem Set 7 Solutions}
\author{Yulwon Rhee (202211342)}
\institute{Department of Computer Science and Engineering, Konkuk University}

\maketitle
\subsubsection{1.} Consider the Matrix $A = \begin{bmatrix}
    1&0&1\\
    -1&2&1\\
    -2&0&4
\end{bmatrix}$

(a) Calculate the eigenvalues and eigenvectors of $A$.
\paragraph*{Solution.}
$$\mathrm{det}(A-\lambda I) = 0$$
\begin{align*}
    \mathrm{det}(A-\lambda I) &= \begin{vmatrix}
        1-\lambda&0&1\\
        -1&2-\lambda&1\\
        -2&0&4-\lambda
    \end{vmatrix}\\
    &= -\lambda^3 + 7\lambda^2-16\lambda+12\\
    &= -(\lambda - 3)(\lambda^2 -4\lambda + 4)\\
    &= -(\lambda - 3)(\lambda - 2)^2\\
    &= 0
\end{align*}
$$\therefore \lambda_1 = 3,\ \lambda_2 = 2$$

(b) If $A$ has three linearly independent eigenvectors, find the diagonalization of $A$

(c) What are the eigenvalues and eigenvectors of $A^{-1}$ (if $A^{-1}$ exists).

(d) Let $B = A + 3I$, where $I$ is the identity matrix. Find the eigenvalues of $B$.

\subsubsection{2.} Consider a square matrix $A$. Suppose that $A$ has full column rank. Can $A$ have eigenvalue $0$? Justify your answer.

\subsubsection{3.} The characteristic polynomial equation (CPE) of $A$ is written as $|A - \lambda I| = 0$. Likewise, the characteristic polynomial equation of $AT$ is $|AT - \lambda I| = 0$. Solving CPE gives eigenvalues.

(a) Show that $A$ and $A^T$ have the same eigenvalues (Hint. Take a look at the CPEs of $A$ and $A^T$, and use the fact that transposing a matrix does not change the determinant)

\subsubsection{4.} Consider a matrix $A = \begin{bmatrix}2&a\\1&0\end{bmatrix}$. Find a condition for $A$ to be diagonalisable.
(Hint. $A$ needs to have two linearly independent eigenvectors. Eigenvectors associated to distinct eigenvalues are linearly independent)

\subsubsection{5.} Check if the following matrices are positive definite:

(a) $\begin{bmatrix}
    2&2&0\\
    2&5&3\\
    0&3&8
\end{bmatrix}$

(b) $\begin{bmatrix}
    1&2\\3&4
\end{bmatrix}$

\subsubsection{6.} Show that $R^TR$ is positive semidefinite for any matrix $R$.

\subsubsection{7.} Show that $R^TR$ is positive definite if and only if $R$ has full column rank.

\subsubsection{8.} Prove that if $B = M^{-1}AM$, then $A$ and $B$ have the same eigenvalues. (Hint: multiply an eigenvector of $B$ on the right, and then multiply $M$ on the left)

\end{document}