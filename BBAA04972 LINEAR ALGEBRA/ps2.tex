% !TeX program = xelatex
\documentclass[runningheads]{llncs}
\usepackage[paperheight=295mm,paperwidth=210mm]{geometry}
\usepackage{graphicx}
\usepackage{wrapfig}
\usepackage{import}
\usepackage{kotex}
\usepackage[dvipsnames]{xcolor}
\usepackage{fancyvrb} %
\usepackage{listings}
\usepackage{tabularx}
\usepackage{underscore}
\usepackage{multicol}
\usepackage{enumitem}
\usepackage{subcaption}
\usepackage[numbers,square,super]{natbib}
\usepackage{mathptmx} % Times New Roman
\usepackage{amsmath}
\usepackage{amssymb}
\usepackage{framed}
\usepackage{etoolbox}
\usepackage{cancel}
\usepackage{tikz}
\usepackage{parskip}
\usepackage{enumerate}
\usepackage{minted}
\usepackage{inconsolata}
\usepackage{makecell}
\usepackage{slashed}
\usepackage{nicematrix}
\usetikzlibrary{calc, angles, quotes, graphs, positioning, arrows, graphs.standard}

\setcounter{tocdepth}{2}

\colorlet{shadecolor}{gray!30}

\newcommand\enclosebox[2]{%
  \BeforeBeginEnvironment{#1}{\begin{#2}}%
  \AfterEndEnvironment{#1}{\end{#2}}%
}

\enclosebox{theorem}{oframed}
\enclosebox{definition}{leftbar}

\newcommand{\divides}{\bigm|}
\newcommand{\ndivides}{%
  \mathrel{\mkern.5mu % small adjustment
    % superimpose \nmid to \big|
    \ooalign{\hidewidth$\big|$\hidewidth\cr$\nmid$\cr}%
  }%
}
\newcommand{\ord}{\operatorname{\mathrm{ord}}}
\newcommand{\ind}{\operatorname{\mathrm{ind}}}
\newcommand{\legendre}[2]{\left(\frac{#1}{#2}\right)}
\setmainfont{Times New Roman}
\setmainhangulfont{KoPubWorldBatang_Pro}
\setmonofont{SFMono Nerd Font}
% \setlength{\parindent}{1em}
% \setlength{\parskip}{1em}
\linespread{1.2}
\renewcommand{\arraystretch}{1.5}
\setlength{\tabcolsep}{0.5em}%
\newenvironment{Figure}
{\par\medskip\noindent\minipage{\linewidth}}
{\endminipage\par\medskip}
\newcommand{\translation}[1]{\textsuperscript{#1}}

\makeatletter
\renewcommand\NAT@citesuper[3]{\ifNAT@swa
  \if*#2*\else#2\NAT@spacechar\fi
  \unskip\kern\p@\textsuperscript{\NAT@@open#1\if*#3*\else,\NAT@spacechar#3\fi\NAT@@close}%
  \else #1\fi\endgroup}
\makeatother

\let\oldtabular\tabular% Store a copy of \tabular
\let\endoldtabular\endtabular% Store a copy of \endtabular
\renewenvironment{tabular}[2][\arraystretch]
{\edef\arraystretch{#1}% Update \arraystretch
  \oldtabular{#2}}% \begin{tabular}[<stretch>]{<col spec>}
{\endoldtabular}% \end{tabular}

\setminted{linenos, fontsize=\small, breaklines}

\begin{document}

\title{Linear Algebra (0031)\newline\space Problem Set 2 Solutions}
\author{Yulwon Rhee (202211342)}
\institute{Department of Computer Science and Engineering, Konkuk University}

\maketitle
\subsubsection{1.}
(a) Find the $3 \times 3$ elimination matrix $E_{21}$ that subtracts $2$ times row $1$ from row $2$.
\paragraph*{Solution.}
$$E_{21} = \begin{bmatrix}
    1&0&0\\
    -2&1&0\\
    0&0&1
\end{bmatrix}$$\\

(b) Find the $3 \times 3$ elimination matrix $E_{31}$ that subtracts $-3$ times row $1$ from row $3$.
\paragraph*{Solution.}
$$E_{31} = \begin{bmatrix}
    1&0&0\\
    0&1&0\\
    -3&0&1
\end{bmatrix}$$\\

(c) Calculate $E_1=E_{31}E_{21}$. What does $E_1$ do in terms of elimination?
Based on your interpretation of what $E_1$ does, calculate the inverse of $E_1$.
\paragraph*{Solution.}
\begin{align*}
    E_1 &= E_{31}E_{21}\\
    &=\begin{bmatrix}
        1&0&0\\
        0&1&0\\
        -3&0&1
    \end{bmatrix} \begin{bmatrix}
        1&0&0\\
        -2&1&0\\
        0&0&1
    \end{bmatrix}\\
    &=\begin{bmatrix}
        1&0&0\\
        -2&1&0\\
        -3&0&1
    \end{bmatrix}
\end{align*}
My interpretation of $E_1$ is to subtract $2$ times row $1$ from row $2$ and subtract $-3$ times row $1$ from row $3$.
So, inverse of $E_1$ should add $2$ times row $1$ from row $2$ and add $-3$ times row $1$ from row $3$.
\begin{align*}
    \therefore E_1^{-1} &= \begin{bmatrix}
        1&0&0\\
        2&1&0\\
        3&0&1
    \end{bmatrix}
\end{align*}
\newpage
(d) Find the $3\times 3$ elimination matrix $E_{32}$ that subtracts $5$ times row $2$ from row $3$.
\paragraph*{Solution.}
$$E_{32} = \begin{bmatrix}
    1&0&0\\0&1&0\\0&-5&1
\end{bmatrix}$$\\

(e) Calculate the inverse of $E_{32}$.
\paragraph*{Solution.}
As the $E_{32}$ subtracts $5$ times row $2$ from row $3$, inverse of $E_{32}$ should add $5$ times row $2$ from row $3$.
$$\therefore E_{32}^{-1}=\begin{bmatrix}
    1&0&0\\0&1&0\\0&5&1
\end{bmatrix}$$\\

(f) Calculate $E=E_{32}E_{31}E_{21}$ and interpret what $E$ does in terms of elimination.
\paragraph*{Solution.}
\begin{align*}
    E&=E_{32}E_{31}E_{21}\\
    &=E_{32}E_1 (\because \text{(c) and Associative Property})\\
    &=\begin{bmatrix}
        1&0&0\\0&1&0\\0&-5&1
    \end{bmatrix}\begin{bmatrix}
        1&0&0\\-2&1&0\\-3&0&1
    \end{bmatrix}\\
    &=\begin{bmatrix}
        1&0&0\\-2&1&0\\7&-5&1
    \end{bmatrix}
\end{align*}
My interpretation of $E$ is to subtract $2$ times row $1$ from row $2$, add $7$ times row $1$ from row $3$ and subtract $5$ times row $2$ from row $3$.\\

\newpage
(g) In order to calculate the inverse of $E$, use the following two methods:\\
\begin{itemize}
    \item Use the properties of inverse matrix as $E^{-1}=(E_{32}E_{31}E_{21})^{-1}=E_{21}^{-1}E_{31}^{-1}E_{32}^{-1}$\\
    \paragraph*{Solution.}
    \begin{align*}
        E^{-1} &= (E_{32}E_{31}E_{21})^{-1}\\
        &= E_{21}^{-1}E_{31}^{-1}E_{32}^{-1}\\
        &= \begin{bmatrix}
            1&0&0\\2&1&0\\0&0&1
        \end{bmatrix} \begin{bmatrix}
            1&0&0\\0&1&0\\3&0&1
        \end{bmatrix}\begin{bmatrix}
            1&0&0\\0&1&0\\0&5&1
        \end{bmatrix}\\
        &= \begin{bmatrix}
            1&0&0\\2&1&0\\3&0&1
        \end{bmatrix} \begin{bmatrix}
            1&0&0\\0&1&0\\0&5&1
        \end{bmatrix}\\
        &= \begin{bmatrix}
            1&0&0\\
            2&1&0\\
            3&5&1
        \end{bmatrix}
    \end{align*}
    \item Use the interpretation of what $E$ does, and derive the inverse that reverts what $E$ does.\\
    \paragraph*{Solution.}
    As the interpretation of matrix $E$,
    $$\therefore E^{-1} = \begin{bmatrix}
            1&0&0\\
            2&1&0\\
            3&5&1
        \end{bmatrix}$$
\end{itemize}

Do the results in i and ii coincide? Yes.
\newpage
\subsubsection{2.}
\paragraph*{Solution.}
\begin{align*}
    LU&=A\\
    &=\begin{bmatrix}
        1&0&0\\
        2&1&0\\
        -2&3&1
    \end{bmatrix}\begin{bmatrix}
        1&1&1\\0&1&1\\0&0&1
    \end{bmatrix}\\
    &=\begin{bmatrix}
        1&1&1\\2&3&3\\-2&1&2
    \end{bmatrix}
\end{align*}
\begin{align*}
    &\left[\begin{array}{ccc|c}
        1&1&1&1\\
        2&3&3&2\\
        -2&1&2&3
    \end{array}\right]
    \\&=\left[\begin{array}{ccc|c}
        1&1&1&1\\
        0&1&1&0\\
        -2&1&2&3
    \end{array}\right]
    \\&=\left[\begin{array}{ccc|c}
        1&1&1&1\\
        0&1&1&0\\
        0&3&4&5
    \end{array}\right]
    \\&=\left[\begin{array}{ccc|c}
        1&1&1&1\\
        0&1&1&0\\
        0&0&1&5
    \end{array}\right]
\end{align*}
$$\therefore\mathbf{x}=\begin{bmatrix}
    1\\-5\\5
\end{bmatrix}$$
\newpage
\subsubsection{3.}
\paragraph*{Solution.}
Let $A = \begin{bmatrix}
    1&\alpha\\\alpha & \beta
\end{bmatrix} (\because \text{(a)})$.

$$\begin{bmatrix}
    1&\alpha\\\alpha & \beta
\end{bmatrix} \Rightarrow \begin{bmatrix}
    1&\alpha\\0&\beta-\alpha^2
\end{bmatrix}$$
$$\therefore\beta-\alpha^2=\frac{5}{2} (\because \text{(a)})$$
$$\therefore 1+2\alpha+\beta=15 (\because \text{(b)})$$
$$\begin{cases}
    \beta-\alpha^2=\frac{5}{2}\\
    2\alpha+\beta=14
\end{cases}$$
$$\therefore \alpha = -1-\frac{5}{\sqrt{2}}, \beta = 16 + 5\sqrt{2} \text{  or  } \alpha = \frac{5}{\sqrt{2}}-1, \beta = 16 - 5\sqrt{2}$$
$$\therefore A = \begin{bmatrix}
    1&-1-\frac{5}{\sqrt{2}}\\-1-\frac{5}{\sqrt{2}}&16 + 5\sqrt{2}
\end{bmatrix} \text{  or  } \begin{bmatrix}
    1&\frac{5}{\sqrt{2}}-1\\\frac{5}{\sqrt{2}}-1&16 - 5\sqrt{2}
\end{bmatrix}$$
\newpage
\subsubsection{4.}
(a) Apply Gaussian elimination to solve $A\mathbf{x} = \mathbf{b}$.
\paragraph*{Solution.}
\begin{align*}
    &\left[\begin{array}{ccc|c}
        1&1&1&9\\
        3&4&4&32\\
        6&8&9&67
    \end{array}\right]\\
    &=\left[\begin{array}{ccc|c}
        1&1&1&9\\
        0&1&1&5\\
        6&8&9&67
    \end{array}\right]\\
    &=\left[\begin{array}{ccc|c}
        1&1&1&9\\
        0&1&1&5\\
        0&2&3&13
    \end{array}\right]\\
    &=\left[\begin{array}{ccc|c}
        1&1&1&9\\
        0&1&1&5\\
        0&0&1&3
    \end{array}\right]
\end{align*}
$$\therefore\mathbf{x}=\begin{bmatrix}
    4\\2\\3
\end{bmatrix}$$\\

(b) Let $P=\begin{bmatrix}
    0&0&1\\1&0&0\\0&1&0
\end{bmatrix}$, $B=PA$ and $\mathbf{c}=P\mathbf{b}$. Find a solution to $B\mathbf{x}=\mathbf{c}$ "without" solving the equation directly.\\
First, $P$ shifts row $1$ to row $2$, row $2$ to row $3$ and row $3$ to row $1$.
$$\therefore B = \begin{bmatrix}
    6&8&9\\
    1&1&1\\
    3&4&4
\end{bmatrix}$$
and,
$$\therefore \mathbf{c} = \begin{bmatrix}
    67\\9\\32
\end{bmatrix}$$
Therefore, $B\mathbf{x}=\mathbf{c}$ is same with row exchanged $A\mathbf{x} = \mathbf{b}$.
As row exchange does not change the value of $\mathbf{x}$,
$$\therefore \mathbf{x} = \begin{bmatrix}
    4\\2\\3
\end{bmatrix}$$
\newpage
\subsubsection{5.}
(a) $\begin{bmatrix}
    2&2\\14&17
\end{bmatrix}$
\paragraph*{Solution.}
Subtract 7 times row $1$ from row $2$.
$$\therefore U = \begin{bmatrix}
    2&2\\0&3
\end{bmatrix}$$
$$\therefore L = \begin{bmatrix}
    1&0\\7&1
\end{bmatrix}$$
And the pivots are $2$ and $3$.\\
\begin{align*}
    &\left[\begin{array}{cc|cc}
        2&2&1&0\\14&17&0&1
    \end{array}\right]\\
    &\Rightarrow\left[\begin{array}{cc|cc}
        1&1&\frac{1}{2}&0\\14&17&0&1
    \end{array}\right]\\
    &\Rightarrow\left[\begin{array}{cc|cc}
        1&1&\frac{1}{2}&0\\
        0&3&-7&1
    \end{array}\right]\\
    &\Rightarrow\left[\begin{array}{cc|cc}
        1&1&\frac{1}{2}&0\\
        0&1&-\frac{7}{3}&\frac{1}{3}
    \end{array}\right]\\
    &\Rightarrow\left[\begin{array}{cc|cc}
        1&0&\frac{17}{6}&-\frac{1}{3}\\
        0&1&-\frac{7}{3}&\frac{1}{3}
    \end{array}\right]
\end{align*}
Therefore, inverse matrix is $\begin{bmatrix}
    \frac{17}{6}&-\frac{1}{3}\\
    -\frac{7}{3}&\frac{1}{3}
\end{bmatrix}$\\
\newpage
(b) $\begin{bmatrix}
    2&-1&1\\4&1&1\\2&5&3
\end{bmatrix}$
\paragraph*{Solution.}
Let $L = \begin{bmatrix}
    1&0&0\\0&1&0\\0&0&1
\end{bmatrix}$.
Subtract $2$ times row $1$ from row $2$.
$$\begin{bmatrix}
    2&-1&1\\0&3&-1\\2&5&3
\end{bmatrix}$$
$$l_{21} = 2$$
Subtract row $1$ from row $3$.
$$\begin{bmatrix}
    2&-1&1\\0&3&-1\\0&6&2
\end{bmatrix}$$
$$l_{31} = 1$$
Subtract $2$ times row $2$ from row $3$.
$$\begin{bmatrix}
    2&-1&1\\0&3&-1\\0&0&4
\end{bmatrix}$$
$$l_{32} = 2$$
$$\therefore L = \begin{bmatrix}
    1&0&0\\2&1&0\\1&2&1
\end{bmatrix}, U = \begin{bmatrix}
    2&-1&1\\0&3&-1\\0&0&4
\end{bmatrix}$$
And the pivots are $2, 3$ and $4$.
\begin{align*}
    &\left[\begin{array}{ccc|ccc}
        2&-1&1&1&0&0\\4&1&1&0&1&0\\2&5&3&0&0&1
    \end{array}\right]\\
    &\Rightarrow\left[\begin{array}{ccc|ccc}
        1&-\frac{1}{2}&\frac{1}{2}&\frac{1}{2}&0&0\\
        4&1&1&0&1&0\\
        2&5&3&0&0&1
    \end{array}\right]\\
    &\Rightarrow\left[\begin{array}{ccc|ccc}
        1&-\frac{1}{2}&\frac{1}{2}&\frac{1}{2}&0&0\\
        0&3&-1&-2&1&0\\
        0&6&2&-1&0&1
    \end{array}\right]\\
    &\Rightarrow\left[\begin{array}{ccc|ccc}
        1&-\frac{1}{2}&\frac{1}{2}&\frac{1}{2}&0&0\\
        0&1&-\frac{1}{3}&-\frac{2}{3}&\frac{1}{3}&0\\
        0&6&2&-1&0&1
    \end{array}\right]\\
    &\Rightarrow\left[\begin{array}{ccc|ccc}
        1&0&\frac{1}{3}&\frac{1}{6}&\frac{1}{6}&0\\
        0&1&-\frac{1}{3}&-\frac{2}{3}&\frac{1}{3}&0\\
        0&0&4&3&-2&1
    \end{array}\right]\\
    &\Rightarrow\left[\begin{array}{ccc|ccc}
        1&0&\frac{1}{3}&\frac{1}{6}&\frac{1}{6}&0\\
        0&1&-\frac{1}{3}&-\frac{2}{3}&\frac{1}{3}&0\\
        0&0&1&\frac{3}{4}&-\frac{1}{2}&\frac{1}{4}
    \end{array}\right]\\
    &\Rightarrow\left[\begin{array}{ccc|ccc}
        1&0&0&-\frac{1}{12}&\frac{1}{3}&-\frac{1}{12}\\
        0&1&0&-\frac{5}{12}&\frac{1}{6}&\frac{1}{12}\\
        0&0&1&\frac{3}{4}&-\frac{1}{2}&\frac{1}{4}
    \end{array}\right]\\
\end{align*}
Therefore, inverse matrix is $\begin{bmatrix}
    -\frac{1}{12}&\frac{1}{3}&-\frac{1}{12}\\
    -\frac{5}{12}&\frac{1}{6}&\frac{1}{12}\\
    \frac{3}{4}&-\frac{1}{2}&\frac{1}{4}
\end{bmatrix}$
\newpage
\subsubsection{6.}
(a) Is it possible for $A$ to have only one pivot? If so, find the values of $a, b, c, d$ so that $A$ has only one pivot
\paragraph*{Solution.}
\begin{align*}
    \begin{bmatrix}
        1&1&1\\a&1&1\\2&b&2\\d&1&c
    \end{bmatrix} \Rightarrow \begin{bmatrix}
        1&1&1\\0&1-a&1-a\\0&b-2&0\\0&1-d&c-d
    \end{bmatrix}
\end{align*}
To $A$ have only one pivot, $a = 1, b = 2, c= 1, d = 1$.\\

(b) Given $d = 2$, let $k$ be the minimum possible number of pivots $A$ can have. Find the condition for $A$ to have exactly $k$ pivots.
\paragraph*{Solution.}
\begin{align*}
    \begin{bmatrix}
        1&1&1\\a&1&1\\2&b&2\\2&1&c
    \end{bmatrix} \Rightarrow \begin{bmatrix}
        1&1&1\\0&1-a&1-a\\0&b-2&0\\0&-1&c-2
    \end{bmatrix} \Rightarrow \begin{bmatrix}
        1&1&1\\0&-1&c-2\\0&1-a&1-a\\0&b-2&0
    \end{bmatrix}
\end{align*}
If $a = 1, b = 2$, the minimum $k = 2$.\\

(c) Given $d = 2$, find the condition (if exists) for $A$ to have exactly $k + 1$ pivots.
\paragraph*{Solution.}
\begin{align*}
    \begin{bmatrix}
        1&1&1\\a&1&1\\2&b&2\\2&1&c
    \end{bmatrix} \Rightarrow \begin{bmatrix}
        1&1&1\\0&-1&c-2\\0&1-a&1-a\\0&b-2&0
    \end{bmatrix} \Rightarrow \begin{bmatrix}
        1&1&1\\0&-1&c-2\\0&0&1-a+(1-a)(c-2)\\0&b-2&0
    \end{bmatrix}
\end{align*}
If $a \neq 1, b = 2$, the value of $k$ will be $3$.

(d) Given $d = 2$, find the condition (if exists) for $A$ to have exactly $k + 2$ pivots.
\paragraph*{Solution.}
It does not exists.\\

(e) Given $d = 2$, find the condition (if exists) for $A$ to have exactly $k + 3$ pivots.
\paragraph*{Solution.}
It does not exists.
\newpage
\subsubsection{7.}
(a)
\paragraph*{Solution.}
\begin{align*}
    \left[\begin{array}{cccc|cccc}
        1&0&0&0&1&0&0&0\\
        2&1&0&0&0&1&0&0\\
        1&0&1&0&0&0&1&0\\
        2&0&0&1&0&0&0&1
    \end{array}\right]\Rightarrow\left[\begin{array}{cccc|cccc}
        1&0&0&0&1&0&0&0\\
        0&1&0&0&-2&1&0&0\\
        0&0&1&0&-1&0&1&0\\
        0&0&0&1&-2&0&0&1
    \end{array}\right]
\end{align*}
The inverse of the matrix is $\begin{bmatrix}
    1&0&0&0\\-2&1&0&0\\-1&0&1&0\\-2&0&0&1
\end{bmatrix}$\\

(b)
\paragraph*{Solution.}
\begin{align*}
    &\left[\begin{array}{ccc|ccc}
        \sin\theta&-\cos\theta&0&1&0&0\\
        \cos\theta&\sin\theta&0&0&1&0\\
        0&0&1&0&0&1
    \end{array}\right]\\
    &\Rightarrow\left[\begin{array}{ccc|ccc}
        \sin\theta&-\cos\theta&0&1&0&0\\
        0&\frac{1}{\sin\theta}&0&-\frac{\cos\theta}{\sin\theta}&1&0\\
        0&0&1&0&0&1
    \end{array}\right]\\
    &\Rightarrow\left[\begin{array}{ccc|ccc}
        \sin\theta&0&0&1-\cos^2\theta&\sin\theta\cos\theta&0\\
        0&\frac{1}{\sin\theta}&0&-\frac{\cos\theta}{\sin\theta}&1&0\\
        0&0&1&0&0&1
    \end{array}\right]\\
    &\Rightarrow\left[\begin{array}{ccc|ccc}
        1&0&0&\sin\theta&\cos\theta&0\\
        0&1&0&-\cos\theta&\sin\theta&0\\
        0&0&1&0&0&1
    \end{array}\right]
\end{align*}
The inverse of the matrix is $\begin{bmatrix}
    \sin\theta&\cos\theta&0\\
    -\cos\theta&\sin\theta&0\\
    0&0&1
\end{bmatrix}$\\
(c)
\paragraph*{Solution.}
\begin{align*}
    &\left[\begin{array}{ccc|ccc}
        0&1&0&1&0&0\\0&0&1&0&1&0\\1&0&0&0&0&1
    \end{array}\right]\\
    &\Rightarrow\left[\begin{array}{ccc|ccc}
        1&0&0&0&0&1\\0&1&0&1&0&0\\0&0&1&0&1&0
    \end{array}\right]
\end{align*}
The inverse of the matrix is $\begin{bmatrix}
    0&0&1\\1&0&0\\0&1&0
\end{bmatrix}$\\

(d)
\paragraph*{Solution.}
\begin{align*}
    &\left[\begin{array}{ccc|ccc}
        a&0&0&1&0&0\\
        0&b&0&0&1&0\\
        0&0&c&0&0&1
    \end{array}\right]\\
    &\Rightarrow\left[\begin{array}{ccc|ccc}
        1&0&0&\frac{1}{a}&0&0\\
        0&1&0&0&\frac{1}{b}&0\\
        0&0&1&0&0&\frac{1}{c}
    \end{array}\right]
\end{align*}
The inverse of the matrix is $\begin{bmatrix}
    \frac{1}{a}&0&0\\
    0&\frac{1}{b}&0\\
    0&0&\frac{1}{c}
\end{bmatrix}$\\

(e)
\paragraph*{Solution.}
\begin{align*}
    &\left[\begin{array}{ccc|ccc}
        1&0&0&1&0&0\\
        2&1&0&0&1&0\\
        3&2&1&0&0&1
    \end{array}\right]\\
    &\Rightarrow\left[\begin{array}{ccc|ccc}
        1&0&0&1&0&0\\
        0&1&0&-2&1&0\\
        3&2&1&0&0&1
    \end{array}\right]\\
    &\Rightarrow\left[\begin{array}{ccc|ccc}
        1&0&0&1&0&0\\
        0&1&0&-2&1&0\\
        0&2&1&-3&0&1
    \end{array}\right]\\
    &\Rightarrow\left[\begin{array}{ccc|ccc}
        1&0&0&1&0&0\\
        0&1&0&-2&1&0\\
        0&0&1&-3&-2&1
    \end{array}\right]
\end{align*}
The inverse of the matrix is $\begin{bmatrix}
    1&0&0\\-2&1&0\\-3&-2&1
\end{bmatrix}$
\newpage
\subsubsection{8.}
\paragraph*{Solution.}
If $P^{-1}$ equals to $P^T$, $PP^T=I$ will hold $(\because PP^{-1}=I)$.
\begin{align*}
    (PP^T)_{ij} &= \sum^n_{k=1}P_{ik}P^T_{kj}\\
    &= \sum^n_{k=1}P_{ik}P_{jk} = \begin{cases} 1 & (i = j) \\ 0 & (i \neq j) \end{cases}
\end{align*}
and the formula above is exactly same with the definition of identity matrix.
$$\therefore PP^T = I$$
$$\therefore P^{-1} = P^T$$
\end{document}